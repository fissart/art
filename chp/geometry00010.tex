\chapter{Fractales}


Un comjunto $E\subset \Real^n$ es autosemejante si existe una coleccion $\xi_1,\xi_2 \cdots \xi_m$ de semejansas de $\Real^n,$ todoas ellas con razones menores a la unidad (es decri contractivas), tales que

\begin{itemize}
  \item $E\subset \bigcup^m_{i=1}\xi_i\pa{E}$
  \item para cierrto $s$ (no necesariamente entero) se tiene ue $H^s\pa{E}>0$ y que $H^s\pa{\xi_i\pa{E}\cap \xi\pa{E}}=0,$ si $i\neq j$
\end{itemize}

Las estructuras de las superficies que observamos no son lo que parecen con una finitud limitada mas aun se puede sumergir al infinito atómicamente y averiguar mentalmente la composición estructural de tales formas, desde  la concepción de este tema e podrá cambiar el modo de ver de las cosas .

Los hay e
%\pspicture(-2,-2)(2,2)\psframe(-2,-2)(2,2)
%\psfractal[type=Mandel, xWidth=6cm,
% yWidth=4.8cm, baseColor=white,
% dIter=10](-2,-1.2)(1,1.2)
%\endpspicture
\begin{figure}
	 \centering
	 \begin{asy}
	 size(200);
	 import palette;
	 import three;

	 currentprojection=perspective(1,1,1);

	 triple[] M=
	 {
	 (-1,-1,-1),(0,-1,-1),(1,-1,-1),(1,0,-1),
	 (1,1,-1),(0,1,-1),(-1,1,-1),(-1,0,-1),
	 (-1,-1,0),(1,-1,0),(1,1,0),(-1,1,0),
	 (-1,-1,1),(0,-1,1),(1,-1,1),(1,0,1),(1,1,1),(0,1,1),(-1,1,1),(-1,0,1)
	 };

	 surface[] Squares=
	 {
	 surface((1,-1,-1)--(1,1,-1)--(1,1,1)--(1,-1,1)--cycle),
	 surface((-1,-1,-1)--(-1,1,-1)--(-1,1,1)--(-1,-1,1)--cycle),
	 surface((1,1,-1)--(-1,1,-1)--(-1,1,1)--(1,1,1)--cycle),
	 surface((1,-1,-1)--(-1,-1,-1)--(-1,-1,1)--(1,-1,1)--cycle),
	 surface((1,-1,1)--(1,1,1)--(-1,1,1)--(-1,-1,1)--cycle),
	 surface((1,-1,-1)--(1,1,-1)--(-1,1,-1)--(-1,-1,-1)--cycle),
	 };

	 int[][] SquaresPoints=
	 {
	 {2,3,4,10,16,15,14,9},
	 {0,7,6,11,18,19,12,8},
	 {4,5,6,11,18,17,16,10},
	 {2,1,0,8,12,13,14,9},
	 {12,13,14,15,16,17,18,19},
	 {0,1,2,3,4,5,6,7}
	 };

	 int[][] index=
	 {
	 {0,2,4},{0,1},{1,2,4},{2,3},{1,3,4},{0,1},{0,3,4},{2,3},
	 {4,5},{4,5},{4,5},{4,5},
	 {0,2,5},{0,1},{1,2,5},{2,3},{1,3,5},{0,1},{0,3,5},{2,3}
	 };

	 int[] Sponge0=array(n=6,value=1);

	 int[] eraseFaces(int n, int[] Sponge0) {
	 int[] temp=copy(Sponge0);
	 for(int k : index[n]) {
	 temp[k]=0;
	 }
	 return temp;
	 }

	 int[][] Sponge1=new int[20][];
	 for(int n=0; n < 20; ++n) {
	 Sponge1[n]=eraseFaces(n,Sponge0);
	 }

	 int[][] eraseFaces(int n, int[][] Sponge1) {
	 int[][] temp=copy(Sponge1);
	 for(int k : index[n])
	 for(int n1 : SquaresPoints[k])
	 temp[n1][k]=0;
	 return temp;
	 }

	 int[][][] Sponge2=new int[20][][];
	 for(int n=0; n < 20; ++n)
	 Sponge2[n]=eraseFaces(n,Sponge1);

	 int[][][] eraseFaces(int n, int[][][] Sponge2) {
	 int[][][] temp=copy(Sponge2);
	 for(int k : index[n])
	 for(int n2: SquaresPoints[k])
	 for(int n1: SquaresPoints[k])
	 temp[n2][n1][k]=0;
	 return temp;
	 }

	 int[][][][] Sponge3=new int[20][][][];
	 for(int n=0; n < 20; ++n)
	 Sponge3[n]=eraseFaces(n,Sponge2);

	 surface s3;
	 real u=2/3;
	 for(int n3=0; n3 < 20; ++n3) {
	 surface s2;
	 for(int n2=0; n2 < 20; ++n2) {
	 surface s1;
	 for(int n1=0; n1 < 20; ++n1) {
	 for(int k=0; k < 6; ++k) {
	 if(Sponge3[n3][n2][n1][k] > 0) {
	 s1.append(scale3(u)*shift(M[n1])*scale3(0.5)*Squares[k]);
	 }
	 }
	 }
	 s2.append(scale3(u)*shift(M[n2])*scale3(0.5)*s1);
	 }
	 s3.append(scale3(u)*shift(M[n3])*scale3(0.5)*s2);
	 }
	 s3.colors(palette(s3.map(abs),Rainbow()));
	 draw(s3);
	 \end{asy}

\caption{wwwww}
\end{figure}

\subsection{Phylotaxis}
The beautiful arrangement de las hojas en alguna plantas, llamado phyllotaxis, obeys un munero de subtle mathematical relationships. For instance, the florets in the head of a sunflower form two oppositely directed spirals: 55 of them clockwise and 34 counterclockwise.Surprisingly, these numbers are consecutive Fibonacci numbers. The Phyllotaxis is like a Lindenmayer system.


\begin{figure}
\begin{center}
\begin{pspicture}[showgrid=true](-3,-3)(3,3)
\psPhyllotaxis[c=4,angle=111]
\end{pspicture}
\,
\begin{pspicture}[showgrid=true](-2.5,-2.5)(2.5,2.5)
\psPhyllotaxis[angle=99]
\end{pspicture}
\end{center}
  \caption{e}
\end{figure}


\section{Cosh y sierpinsiqui}

\begin{figure}
\begin{center}
\begin{pspicture}[showgrid=true](0,0)(13,3.7)
 \multido{\iA=0+1,\iB=0+2}{6}{%
 \psKochflake[angle=-30,scale=3,maxIter=\iA](\iB,2.5)%\psdot*(\iB,2.5)
 \psKochflake[scale=3,maxIter=\iA](\iB,0)}%\psdot*(\iB,0)
\end{pspicture}
\end{center}
  \caption{s}\label{s}
\end{figure}

\begin{figure}
\begin{center}
\begin{pspicture}[showgrid=true](-1,0)(4,5)
\psKochflake[scale=10,linewidth=1pt]
\end{pspicture}
\end{center}
  \caption{copo de nieva de Cosh}
\end{figure}






\subsection{Arboles}

\begin{figure}
\begin{center}
\begin{pspicture}[showgrid=true](-7,-1)(5,8)
\psPTree[xWidth=1.75cm,c=0.35]
\end{pspicture}
\end{center}
  \caption{Arbol}\label{f}
\end{figure}

\begin{figure}
\begin{center}
\begin{pspicture}[showgrid=true](-3,0)(3,3.5)
\psFArrow[linewidth=9pt]{0.65}
\end{pspicture}
\,
\begin{pspicture}[showgrid=true](-4,-3)(3,3)\psset{unit=0.7}
\psFArrow[Color]{0.7}
\psFArrow[angle=90,Color]{0.7}
\end{pspicture}
\end{center}
  \caption{e}
\end{figure}

\begin{figure}
\begin{center}
\begin{pspicture}[showgrid=true](0,0)(5,4.5)
\psSier[linecolor=black,](0,0){5cm}{8}
\end{pspicture}
\end{center}
  \caption{triangulo de Sierpinski}
\end{figure}


\subsection{Circulo de Apollonius}

\begin{figure}
\begin{center}
\begin{pspicture}[showgrid=true,linewidth=9pt](-4,-4)(4,4)
 \psAppolonius[Radius=4cm]
\end{pspicture}
\end{center}
  \caption{Circulo de Apollonius}
\end{figure}

\begin{figure}
\begin{center}
\begin{pspicture}[showgrid=true,linewidth=9pt](-5,-5)(5,5)
\psAppolonius[Radius=5cm,Color]
\end{pspicture}
\end{center}
  \caption{Circulo de Apollonius}
\end{figure}





\index{wwwwwwwwwwwwwwww}
