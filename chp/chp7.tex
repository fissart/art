

\chapter{Proporcion y canon}


\section{Proporcion en el arte}
\section{Proporcion directa}
\section{Proporcion inversa}
\section{Canon}

\subsection{Canon griego}
\subsection{Canon egipcio}
\subsection{Canon andino}
El buen gusto por la perfección exquisitas llevo a los generadores de formas reales o abstractas a concebir cañones y reglas, la analogía de la sección áurea con muchas areas de la ciencia no explica como el numero del promedio entre el caos y el orden los rectángulos estáticos'' no producen divisiones armónicas, no obstante los rectángulos dinámicos producen las mas variadas y satisfactorias subdivisiones  y combinaciones distintas sin encontrar antagonismos entre ellos mas aun viéndolas unirse entre ellas para genera un solo objeto bidimensionales  por ejemplo el $\sqrt{5}$ se compone de muchos de el mismo y del rectángulo $\phi$
















\index{wwwwwwwwwwwwwwww}
