\chapter{Geometría de las formas}

\begin{figure}
\centering
\begin{asy}
size(300,0);
import markers;
import geometry;
import math;

pair A=0, B=(1,0), C=(0.7,1), D=(-0.5,0), F=rotate(-90)*(C-B)/2+B;

draw(A--B);
draw(A--C);
pen p=linewidth(1mm);
draw(B--C,p);
draw(A--D);
draw(B--F,p);
label("$A$",A,SW);
label("$B$",B,S);
label("$C$",C,N);
dot(Label("$D$",D,S));
dot(Label("$F$",F,N+NW));

markangle(A,C,B);

markangle(scale(1.5)*"$\theta$",radius=40,C,B,A,ArcArrow(2mm),1mm+blue);
markangle(scale(1.5)*"$-\theta$",radius=-70,A,B,C,ArcArrow,green);

markangle(Label("$\gamma$",Relative(0.25)),n=2,radius=-30,A,C,B,p=0.7blue+2);

markangle(n=3,B,A,C,marker(markinterval(stickframe(n=2),true)));

pen RedPen=0.7red+1bp;
markangle(C,A,D,RedPen,marker(markinterval(2,stickframe(3,4mm,RedPen),true)));
drawline(A,A+dir(A--D,A--C),dotted);

perpendicular(B,NE,F-B,size=10mm,1mm+red,
TrueMargin(linewidth(p)/2,linewidth(p)/2),Fill(yellow));
\end{asy}
\caption{geometry}
\end{figure}

\begin{figure}
\centering
\begin{asy}
size(300,0);
import markers;
import geometry;
import math;

pair A=0, B=(1,0), C=(0.7,1), D=(-0.5,0), F=rotate(-90)*(C-B)/2+B;

draw(A--B);
draw(A--C);
pen p=linewidth(1mm);
draw(B--C,p);
draw(A--D);
draw(B--F,p);
label("$A$",A,SW);
label("$B$",B,S);
label("$C$",C,N);
dot(Label("$D$",D,S));
dot(Label("$F$",F,N+NW));

markangle(A,C,B);

markangle(scale(1.5)*"$\theta$",radius=40,C,B,A,ArcArrow(2mm),1mm+blue);
markangle(scale(1.5)*"$-\theta$",radius=-70,A,B,C,ArcArrow,green);

markangle(Label("$\gamma$",Relative(0.25)),n=2,radius=-30,A,C,B,p=0.7blue+2);

markangle(n=3,B,A,C,marker(markinterval(stickframe(n=2),true)));

pen RedPen=0.7red+1bp;
markangle(C,A,D,RedPen,marker(markinterval(2,stickframe(3,4mm,RedPen),true)));
drawline(A,A+dir(A--D,A--C),dotted);

perpendicular(B,NE,F-B,size=10mm,1mm+red,
TrueMargin(linewidth(p)/2,linewidth(p)/2),Fill(yellow));
\end{asy}
\caption{geometry}
\end{figure}

\begin{figure}
\begin{asy}
import solids;
import  obj;
size3(200,0);
size(300);
currentprojection=perspective(3,3,-1);
currentlight=Headlamp;
triple t=(0,3 ,0);
triple vectaxe=(1,0,0);

triple pA=(-3,0,0), pB=(0,3,0), pC=(0,0,3), pE=(0,0,3);
transform3 sym=reflect(X,Z,O);
transform3 r=rotate(0,vectaxe);
triple p1=(1.3125,-0.531249,0.054691);
triple p2=(0.328125,0.398431,-0.914065);
draw(p1--shift(t)*p1,blue+dashed, Arrow3);
draw(p2--shift(t)*p2,blue+dashed, Arrow3);
dot(p1^^p2^^shift(t)*p1^^shift(t)*p2);

dot(Label("$A$",align=dir(-90)),p1);
dot(Label("$C$",align=dir(-90)),p2);
dot(Label("$B$",align=dir(-90)),shift(t)*p1);
dot(Label("$D$",align=dir(-90)),shift(t)*p2);

draw(obj("www.obj",paleyellow+opacity(0.9)));
draw(shift(t)*obj("www.obj",paleblue+opacity(0.9)));
axes3("$x$","$y$","$z$", Arrows3);
\end{asy}
\caption{objet}
\end{figure}

If the optional boolean argument check is false, no check will be made that the file exists. If the file does not exist or is not readable, the function bool error(file) will return true. The first character of the string comment specifies a comment character. If this character is encountered in a data file, the remainder of the line is ignored. When reading strings, a comment character followed immediately by another comment character is treated as a single literal comment character. If Asymptote is compiled with support for libcurl, name can be a URL.

Unless the -noglobalread command-line option is specified, one can change the current working directory for read operations to the contents of the string s with the function string cd(string s), which returns the new working directory. If string s is empty, the path is reset to the value it had at program startup.

When reading pairs, the enclosing parenthesis are optional. Strings are also read by assignment, by reading characters up to but not including a newline. In addition, Asymptote provides the function string getc(file) to read the next character (treating the comment character as an ordinary character) and return it as a string.

\begin{figure}
\begin{asy}
size(300,0);
import graph3;
currentprojection =perspective(3,4,2);
pair a=(-2,-1);
pair b=(1,1.5);

real f(pair xy) {
real x = xy.x; real y = xy.y;
return (6/5 - x^2/2) * (-y^4/2 + y^3/15 + y^2 + y/5 + 1);
}
real f1(pair xy) {
real x = xy.x; real y = xy.y;
return -x * (-y^4/2 + y^3/15 + y^2 + y/5 + 1);
}
real f2(pair xy) {
real x = xy.x; real y = xy.y;
return (6/5 - x^2/2) * (-2*y^3 + y^2/5 + 2*y + 1/5);
}
real w=1.1 ;
real x1(real t){return t;}
real y1(real t){return w;}
real z1(real t){pair z=(t,w); return f(z);}

path3 l1=graph(x1,y1,z1,a.x,b.x);

real ww=.4;
real x2(real t){return ww;}
real y2(real t){return t;}
real z2(real t){pair z=(ww,t); return f(z);}

path3 l2=graph(x2,y2,z2,a.y,b.y);

triple Q=(ww,w,f((ww,w)));


draw(l1, orange);
draw(l2, white);
dot(Q, orange);
pair tww=(Q.x,Q.y);
real m1=f1(tww);
path3 tgx=Q-unit((Q.x+1,Q.y,Q.z+m1)-Q)--(Q+unit((Q.x+1,Q.y,Q.z+m1)-Q));
draw(tgx, orange);
pair tww2=(Q.x,Q.y);
real m2=f2(tww2);
path3 tgy=Q-unit((Q.x,Q.y+1,Q.z+m2)-Q)--(Q+unit((Q.x,Q.y+1,Q.z+m2)-Q));
draw(tgy, white);
draw(surface(plane(
  O=Q+unit((Q.x,Q.y+1,Q.z+m2)-Q)+unit((Q.x+1,Q.y,Q.z+m1)-Q),
  -2*unit((Q.x,Q.y+1,Q.z+m2)-Q),
  -2*unit((Q.x+1,Q.y,Q.z+m1)-Q)
  )), blue + opacity(0.7));

surface s = surface(f, a, b, Spline);
draw(s, surfacepen=white);axes3("$x$","$y$","$z$", Arrows3);
\end{asy}
\caption{tangent}
\end{figure}






\begin{figure}
\begin{asy}
import graph3;
import solids;
size(300,0);
currentprojection=perspective(3,3,5);

pen color1=green+opacity(0.25);
pen color2=red;
real alpha=350;

real f(real x) {return 2x^2-x^3;}
pair F(real x) {return (x,f(x));}
triple F3(real x) {return (x,f(x),0);}

ngraph=12;

real x1=0.7476;
real x2=1.7787;
real x3=1.8043;

path[] p={graph(F,x1,x2,Spline),
          graph(F,0.7,x1,Spline)--graph(F,x2,x3,Spline)&cycle,
          graph(F,0,0.7,Spline)--graph(F,x3,2,Spline)};

pen[] pn=new pen[] {color1,color2,color1};

render render=render(compression=0);

for(int i=0; i < p.length; ++i) {
  revolution a=revolution(path3(p[i]),Y,0.5,alpha);
  draw(surface(a),pn[i],render);

  surface s=surface(p[i]--cycle);
  draw(s,pn[i],render);
  draw(rotate(alpha,Y)*s,pn[i],render);
}

draw((4/3,0,0)--F3(4/3),dashed);
xtick("$\frac{4}{3}$",(4/3,0,0));

axes3("$x$","$y$","$z$", Arrows3);

//arrow("$y=2x^2-x^3$",F3(1.6),X+Y,0.75cm,red);
draw(arc(1.1Y,0.3,90,0,7.5,180),Arrow3);
\end{asy}
\caption{Seccion de una superficie}
\end{figure}


\begin{figure}
\begin{asy}
import graph3;
import three;
size3(200,0);
currentprojection=perspective(4,6,-3);
//triple pA=(-3,0,0), pB=(0,3,0), pC=(0,0,3), pE=(0,0,3);
//path3 gg=pA--pB--pC--cycle;
//draw(surface(gg),orange);
triple p1=(3,1,1.5);
transform3 t=shift(p1);
transform3 g=scale(1.5,1.5,1.5);

draw(g*unithemisphere,blue);
draw(surface(t*g*unithemisphere),orange);
draw(surface(t*g*g*unithemisphere),yellow);
draw(surface(t*g*g*g*unithemisphere),yellow);
axes3("$x$","$y$","$z$", Arrows3);
\end{asy}
\caption{Escala}
\end{figure}
