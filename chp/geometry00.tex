
\chapter{Formas Geométricas Sobre el Plano}

\section{Rectángulos dinámicos}


Los rectángulos dinámicos se caracterizan por  tener proporciones no racionales es decir irracionales aquí observamos que los rectángulo $\sqrt{2},\sqrt{3}, \sqrt{5},$ etc. excepto el $\sqrt{4}=2$ que es un numero racional también se observa que a partir de un cuadrado Figura~\ref{descuento} se pueden construir sucesivamente estos rectángulo en algunos  casos obviamente mediante este proceso se podrán hallar rectángulos  no dinámicos

Si se hace un test a un grupo de personas sobre la prioridad que tiene el rectángulo áureo en relación  a otros es muy alta



La principal aplicación esta siempre en el diseño y la presencia en el arte plástico es una característica proporcionado por la geometría derivada de la sección áurea o de una cadena de proporciones relacionadas (este es una noción importante donde será ilustrado después), donde se produce la recurrencia de formas similares, pero la sugestión mencionad arriba es especialmente asociada con la Sección Áurea porque ella posee propiedades muy interesantes con la infinita variedad de progresión geométrica de radio
La principal aplicación esta siempre en el diseño y la presencia en el arte plástico es una característica proporcionado por la geometría derivada de la sección áurea o de una cadena de proporciones relacionadas (este es una noción importante donde será ilustrado después), donde se produce la recurrencia de formas similares, pero la sugestión mencionad arriba es especialmente asociada con la Sección Áurea porque ella posee propiedades muy interesantes con la infinita variedad de progresión geométrica de radio $\phi$


\begin{figure}
\begin{center}
\psset{unit=1.2}
\begin{pspicture}(-0.5,-0.8)(9.2,4.4)
%\psgrid[subgriddiv=1,griddots=10]\psframe(-0.5,-0.8)(9.2,4.6)
\rput{0}{\pstGeonode[PosAngle={-135,-45,90,135}](0,0){O}(4,0){A}(4,4){B}(0,4){C}}
\pstInterLC[PosAngle=-135]{O}{A}{O}{B}{R'}{R}
\pstProjection{C}{B}{R}[K]%%%%% K
\pstInterLC[PosAngle=-90]{A}{R}{O}{K}{S'}{S}
\pstProjection{C}{B}{S}[L]%%%%% L
\pstInterLC[PosAngle=-90]{O}{A}{O}{L}{T'}{T}
\pstProjection{C}{B}{T}[M]%%%%% M
\pstInterLC[PosAngle=-90]{O}{A}{O}{M}{U'}{U}
\pstMiddleAB[PosAngle=135]{O}{A}{D}
\pstInterLC[PosAngle=-90,PointSymbolA=none, PointNameA=]{O}{A}{D}{B}{P''}{P}\pstProjection{C}{B}{P}[P']%%%%%
\pstInterLC[PosAngle=-90,PointSymbolA=none, PointNameA=]{O}{A}{O}{P'}{r''}{r}
\pstProjection{C}{B}{r}[r']%%%%% M

\pstProjection{C}{B}{U}[N]%%%%% M
\pspolygon[](O)(R)(K)(C)%=vlines
\pspolygon[](O)(A)(B)(C)%=vlines
\pspolygon[](O)(S)(L)(C)%=vlines
\pspolygon[](O)(T)(M)(C)%=vlines
\pspolygon[](O)(U)(N)(C)%=vlines
\pspolygon[](O)(P)(P')(C)%=vlines
\pcline[]{}(r)(r')
\pcline[linestyle=dashed]{->}(D)(B)\pcline[linestyle=dashed]{->}(O)(r')
\pcline[offset=0pt,]{->}(O)(B)
\ncput*[nrot=:U]{$\sqrt{2}$}
\pcline[offset=0pt]{->}(O)(K)
\ncput*[nrot=:U]{$\sqrt{3}$}
\pcline[offset=0pt]{->}(O)(L)
\ncput*[nrot=:U]{$\sqrt{4}$}
\pcline[offset=0pt]{->}(O)(M)
\ncput*[nrot=:U]{$\ldots$}
\pcline[offset=0pt]{->}(O)(N)
\pcline[offset=-12pt]{|<->|}(O)(A)
\ncput*{$1$}
\pcline[offset=12pt]{|<->|}(O)(C)
\ncput*{$1$}
\pcline[offset=-25pt]{|<->|}(O)(U)
\ncput*{$\sqrt{5}$}
\pstSegmentMark[SegmentSymbol=MarkHashh]{O}{D}
\pstSegmentMark[SegmentSymbol=MarkHashh]{A}{D}
\pstArcnOAB[arrows=->]{O}{B}{R}
\pstArcnOAB[arrows=->]{O}{K}{S}
\pstArcnOAB[arrows=->]{O}{L}{T}
\pstArcnOAB[arrows=->]{O}{M}{U}
\pstArcnOAB[arrows=->,linestyle=dashed]{D}{B}{P}
\pstArcnOAB[arrows=->,linestyle=dashed]{O}{P'}{r}
 \end{pspicture}
\end{center}
\caption{Rectángulos Dinámicos $\sqrt{2},$ $\phi,$ $\sqrt{3},$ $\sqrt{5},$ ...}
\label{descuento}
\end{figure}

Como el rectángulo $ARKC$ denotado por $\sqrt{2}$, $ASLC$ denotado por $\sqrt{3}$, $ATMC$ denotado por $\sqrt{4}=2$ que no es un rectángulo dinámico, $AUNC$ denotado por $\sqrt{5}$ y los rectángulos relacionado con el numero de oro $ACPP'$ denotado por $\phi$ construido con la ayuda del punto medio $D$ del segmento $OA$ finalmente el rectángulo $Orr'C$ denotado por $\sqrt{\phi}$ son los rectángulos más interesantes para la distribución de los elementos en el espacio bidimensional.



Se descompondrá armónicamente cada uno de estos rectángulos, saber el procedimiento es muy útil para los artistas plásticos sobre para los pintores en sus diversas composiciones bidimensionales, para aquellos que tienen noción tridimensional  se trataran de solidos en el siguiente capitulo.

\begin{comen}\label{com1} un rectángulo esta bien representado por su diagonal y la pendiente de esta en un sistema de ejes coordenados usual. Pues si tratamos de averiguar el tipo de rectángulo  lo que se hace es verificar  la razón de la longitud de su lado mayor y al longitud de su lado menor es decir, la pendiente de la diagonal con respecto aun sistema de ejes coordenados donde el eje las $x$ coincide con el lado mayor es decir en la Figura \ref{Op} la pendiente de la diagonal $AC$ es $\tan{\alpha}=\frac{\overline{CB}}{\overline{AB}}.$

Por ejemplo en la Figura \ref{Up} el rectángulo $A'B'C'D'$ tiene las mismas proporciones que $ABCD$ pues la pendiente de $A'C'$ es la misma que la pendiente de $AC,$ este principio nos ayudara a demostrar algunas propiedades de los rectángulos dinámicos.
\begin{figure}
\begin{center}
\begin{pspicture}(-1.5,-0.4)(5,4.4)
%\psframe(-1.4,-0.4)(5,4.4)\psgrid[subgriddiv=1,griddots=10]
\pstGeonode[CurveType=polygon,unit=1,PosAngle={-110,-90,90,-115}](0,0){A}(4,2){B}(3,4){C}(-1,2){D}
\pstLineAB{A}{C}
\pstMarkAngle[]{B}{A}{C}{$\alpha^\circ$}
\pstTranslation[DistCoef=0.4,PointSymbol=none,PosAngle=180]{A}{B}{B}[x]
\pstTranslation[DistCoef=0.4,PointSymbol=none,PosAngle=180]{A}{D}{D}[y]
\pstLineAB[nodesepA=-.4, nodesepB=-1,arrows=->]{A}{B}
\pstLineAB[nodesepA=-.4, nodesepB=-1,arrows=->]{A}{D}
 \end{pspicture}
\end{center}
\caption{Averiguando el tipo de rectángulo}\label{Op}
\end{figure}


\end{comen}



\begin{comen}
A partir de ahora se se usará la notación $ABCD\sim r, r\in \mathbb{I}$ donde $ABCD$ es un rectángulo y ''$\sim$'' significa ''similar semejante'', muy útil  para denotar que dos rectángulos tiene las mismas proporciones o la misma razón entre las longitudes de sus lados  por ejemplo en la Figura \ref{Up} $A'B'C'D'\sim\frac{B'C'}{A'B'}=k; k\in \mathbb{I}$ o en la Figura \ref{Uk} se tiene que $OABC\sim\frac{AB}{OA}=1.$

\begin{figure}
\begin{center}
\begin{pspicture}(-0.4,-0.4)(6.4,3.5)
%\psframe(-0.4,-0.4)(6.4,3.4)\psgrid[subgriddiv=1,griddots=10]
\pstGeonode[CurveType=polygon,unit=1,PosAngle={-90,-90,90,90}](0,0){A}(6,0){B}(6,3){C}(0,3){D}
\pstGeonode[CurveType=polygon,unit=1,PosAngle={-90,-90,-45,-135}](2,1){A'}(5,1){B'}(5,2.5){C'}(2,2.5){D'}
\pspolygon[](A)(B)(C)(D)%=vlines
\pspolygon[](A')(B')(C')(D')%=
\pstLineAB{A}{C}
 \end{pspicture}
\end{center}
\caption{Cuadrado}\label{Up}
\end{figure}

\end{comen}

\begin{comen}

El siguiente criterio mostrada en la Figura \ref{Upu} se toma el $M=\frac{BC}{2}$ se traza el arco $CB$ centrada en $M$ luego $P$ es la intersección de la diagonal del rectángulo $ABCD$ con éste arco, finalmente $Q$ es la intersección del lado $DC$ con la linea $BP$. Se usara este principio para resumir las demostraciones de las propiedades de los rectángulos dinámicos, se tiene que $ABCD\sim P'BCQ$ pues en $AC\perp BQ$ esto es $\angle{BAC}=\angle{CBQ},$
luego segun el Comentario \ref{com1} se tiene que $ABCD\sim P'BCQ$ y tambien se tiene que $\frac{AB}{CB}=\frac{BC}{QC}\Longleftrightarrow QC=\frac{BC^2}{AB}$ si $BC=1$ se tiene que $QC=\frac{1}{AB}$ por lo que si $AB$ es de la forma $\sqrt{\beta}$ se tiene que $QC=\frac{1}{\sqrt{\beta}}=\frac{\sqrt{\beta}}{\beta}$ es decir $QC=\frac{AB}{\beta},$ como un ejemplo particular se tiene que si $AB=\sqrt{6}\Longrightarrow QC=\frac{AB}{6}.$

 \begin{figure}
\begin{center}
\begin{pspicture}(-0.4,-0.4)(6.4,3.4)
%\psframe(-0.4,-0.4)(6.4,3.4)\psgrid[subgriddiv=1,griddots=10]
\pstGeonode[CurveType=polygon,unit=1,PosAngle={-90,-90,90,90}](0,0){A}(6,0){B}(6,3){C}(0,3){D}
\pstMiddleAB[PosAngle=0]{B}{C}{M}
\pstInterLC[,PointSymbolB=none, PointNameB=]{A}{C}{M}{C}{P}{P''}
\pstInterLL[PosAngle=90]{P}{B}{D}{C}{Q}
\pstProjection[CodeFig=true,CodeFigColor=black]{B}{A}{Q}[P']
\pstArcOAB[linestyle=dashed,arrows=->]{M}{C}{B}
\pstLineAB{B}{Q}
\pstLineAB{A}{C}
\pstRightAngle{C}{P}{Q}
\psset{LabelRefPt=c,arrows=->,MarkAngleRadius=0.6,LabelAngleOffset=0,
LabelSep=1.3}
\pstLineAB[linestyle=dashed,arrows=->]{M}{P}
\pstMarkAngle[]{B}{A}{C}{$\alpha^\circ$}
\pstMarkAngle[]{C}{B}{Q}{$\alpha^\circ$}
 \end{pspicture}
\end{center}
\caption{Un rectángulo arbitrario}\label{Upu}
\end{figure}

\end{comen}



\subsection{El cuadrado}
Para poder particionarlo es necesario hallar la sección áurea en uno de los lados por ejemplo $P$ con el método ya aprendido, a partir de allí se generan infinidad de posibilidades  por ejemplo una de ellas es la que se muestra en la figura siguiente. aunque el cuadrado es considerado menos apto para las composiciones con un poco de subdivisiones armónicas se pueden obtener una buena composición

\begin{figure}
\begin{center}
\begin{pspicture}(-0.4,-0.4)(4.4,4.4)
%\psframe(-0.4,-0.4)(4.4,4.4)\psgrid[subgriddiv=1,griddots=10]
\pstGeonode[CurveType=polygon,unit=1,PosAngle={-135,-45,90,135}](0,0){O}(4,0){A}(4,4){B}(0,4){C}
\pstMiddleAB[PosAngle=135]{O}{A}{D}
\pstInterLC[PosAngle=135,,PointSymbolA=none, PointNameA=]{D}{B}{D}{A}{E'}{E}
\pstInterLC[,PointSymbolB=none, PointNameB=]{A}{B}{B}{E}{P}{P''}
\pstProjection[CodeFig=true,CodeFigColor=black]{C}{O}{P}[P']
\pstInterLL[PosAngle=-90]{O}{B}{P}{P'}{Q}
\pstLineAB{O}{B}
\pstArcOAB[linestyle=dashed,arrows=->]{D}{A}{E}
\pstArcnOAB[linestyle=dashed,arrows=<-]{B}{P}{E}
\pstLineAB[linestyle=dashed]{D}{B}
%\pstInterLC[PosAngle=-90,PointSymbolA=none, PointNameA=none]{O}{A}{D}{B}{P''}{P}
%\pstProjection{C}{B}{P}[P']%%%%% M
%\pspolygon[](O)(A)(B)(C)%=vlines
%\pcline[offset=0pt]{|<*->|*}(O)(B)
%\ncput*[nrot=:U]{ $\sqrt{2}$}
%\pstArcnOAB[arrows=->, arrowscale=2]{O}{B}{R}



 \end{pspicture}
\end{center}
\caption{Cuadrado}\label{Uk}
\end{figure}

EL cuadrado suele ser uno de los formatos menos eficientes debido a su alta simetría pero con particiones adecuadas sobre su superficie se puede lograr grandes objetivos


\subsection{El rectángulo $\sqrt{2}$}

Siendo $M'$ y $M$ puntos medios de $DC$ y $AB$ se observa la propiedad de $DM\perp AC$  pues la pendiente del a recta $DM$ es $-\frac{2}{\sqrt{2}}$ y la pendiente de la recta $AC$ es $\frac{\sqrt{2}}{2}$ lo cual al multiplicar estas pendientes resulta $-1.$

Otra característica es que $ONMB$  es otro rectángulo $\sqrt{2}$ con el lado mayor $ON=MB$ pues $OM=\frac{\sqrt{2}}{2}$ y $NB=\frac{1}{2}$ entonces $\frac{AM}{NB}=\frac{\frac{\sqrt{2}}{2}}{\frac{1}{2}}=\frac{1}{2},$ en este rectángulo también se observa que $MH\perp HB$ pues $MC$ lo secciona a $ON$ en dos segmentos iguales $OP=PN$ lo cual usando el mismo criterio para el aso anterior  se verifica que $MH\perp HB,$ $HH''\perp H''B$ y $N'N\perp HB$ porque estos puntos se obtiene con el mismo procedimiento.

Finalmente se pueden obtener de manera indefinida rectángulos $\sqrt{2}$ tales como $OMNB,$ $HPBN',$ etc. los cuales convergen hacia el vértice $B.$ También se los puede hacer converger hacia los demás vertices $A, D$ y $C$ del mismo modo en el rectángulo $MONB$ se puede iterar los procedimientos anteriores hacia el infinito.

\begin{figure}
\begin{center}
\begin{pspicture}(-0.3,-0.5)(8.1,6.1)
%\psframe(-0.3,-0.5)(8.1,6.1)\psgrid[subgriddiv=1,griddots=10]
\pstGeonode[unit=5.5,PosAngle={-90,90,90,90},PointSymbol={*,none,none,*,none},PointName={default,none,none,default,none}](0,0){A}(1,0){Bb}(1,1){Cc}(0,1){D}(0.5,0){M}
\pstInterLC[PointSymbolA=none,PointNameA=none,PosAngleB=-90]{A}{Bb}{A}{Cc}{FF}{B}
\pstProjection[]{D}{Cc}{B}[C]
\pstMiddleAB[PosAngle=-90]{A}{B}{M}
\pstMiddleAB[PosAngle=90]{D}{C}{M'}
\pstMiddleAB[PosAngle=0]{B}{C}{N}
\pstMiddleAB[PosAngle=-90]{M}{B}{N'}
\pstMiddleAB[PosAngle=0]{B}{N}{P}
\pspolygon(A)(B)(C)(D)
\pstLineAB{D}{M}
\pstLineAB{M}{C}
\pstLineAB{A}{C}
\pstLineAB{D}{B}
\pstLineAB{B}{M'}\pstLineAB{M}{N}
\pstLineAB{N'}{N}
\pstInterLL[PosAngle=-90]{D}{B}{M}{C}{H}
\pstInterLL[PosAngle=0]{A}{C}{M}{D}{H'}
\pstInterLL[PosAngle=70]{M}{N}{B}{M'}{H''}
\pstInterLL[PosAngle=0]{N'}{N}{B}{D}{I}
\pstInterLL[PosAngle=90]{A}{C}{B}{D}{O}
\pstInterLL[PosAngle=-96,PointName=none,PointSymbol=none]{N}{M}{B}{D}{Hhh}
\pstRightAngle{D}{H}{C}
\pstRightAngle{D}{H'}{C}
\pstRightAngle{M}{H''}{M'}
\pstRightAngle{N'}{I}{D}
\pstHomO[HomCoef=0.8,PosAngle=0,PointSymbol=none,PointName=none]{N'}{P}[K]
\pstLineAB[arrows=->]{N'}{K}
\pstSegmentMark[]{M}{N'}
\pstSegmentMark[]{B}{N'}
\pstInterLL[PosAngle=0]{A}{C}{B}{M'}{O'}
\pstSegmentMark[SegmentSymbol=MarkHash]{P}{B}
\pstSegmentMark[SegmentSymbol=MarkHash]{P}{N}
\pspolygon(M)(O)(N)(B)\pspolygon(N')(Hhh)(P)(B)
\pstMiddleAB[PosAngle=0]{A}{D}{E}
\pstArcOAB[linestyle=dashed,arrows=->]{E}{A}{D}
\pstRotation[RotAngle=90,PosAngle=90]{D}{A}[N'']
\pstRotation[RotAngle=-180,PosAngle=90]{M'}{N''}[R]
\pstArcOAB[linestyle=dashed,arrows=->]{D}{A}{N''}
\pstArcOAB[linestyle=dashed,arrows=->]{M'}{R}{N''}
\pstProjection[CodeFig=true, CodeFigColor=black]{A}{B}{R}[R']
\pstLineAB{R'}{C}
\end{pspicture}
\end{center}
  \caption{$\sqrt{2}$}\label{de}
\end{figure}





\subsection{El rectángulo $\sqrt{3}$}


La propiedad de este triángulo es que si lo dividimos en tres franjas verticales iguales tales como $ANOD$,$ONMP$ y $PMBC$ obtenemos otros triángulos semejantes al primero $\sqrt{3}$, como en el caso anterior se uso las pendientes para averiguar si es correcto poner los ángulos rectos donde lo están, luego es posible iterar esta operación al infinito sobre cada uno de los tres rectángulos  obtenidos anteriormente para obtener otros con la misma propiedad pero en escala menor.


\begin{figure}
\begin{center}
\begin{pspicture}(-0.2,-0.6)(9.2,5.6)
%\psframe(-0.2,-0.6)(9.2,5.6)\psgrid[subgriddiv=1,griddots=10]
\pstGeonode[unit=5,PosAngle={-90,135,-45,90},PointSymbol={*,none,none,*},PointName={default,none,none,default}](0,0){A}(1,0){Bb}(1,1){Cc}(0,1){D}
\pstInterLC[PointSymbolA=none,PointNameA=,PointSymbolB=none,PointNameB=]{A}{Bb}{A}{Cc}{G'}{G}
\pstInterLC[PointSymbolA=none,PointNameA=,PosAngle=90]{D}{Cc}{D}{G}{h'}{C}
\pstProjection[]{A}{Bb}{C}[B]
\pspolygon(A)(B)(C)(D)
\pstLineAB{A}{C}\pstLineAB{B}{D}
\pstProjection[PosAngle=-90,CodeFig=true,CodeFigColor=black]{D}{B}{C}[G]
\pstInterLL[PosAngle=-90]{A}{B}{C}{G}{M}
\pstProjection[PosAngle=180,CodeFig=true,CodeFigColor=black]{A}{C}{D}[H]
\pstInterLL[PosAngle=-90]{A}{B}{D}{H}{N}
\pstProjection[CodeFig=true,CodeFigColor=black]{D}{C}{N}[O]
\pstProjection[CodeFig=true,CodeFigColor=black]{D}{C}{M}[P]
\pstLineAB{O}{N}\pstLineAB{P}{M}
\pstHomO[HomCoef=0.2,PointName=none,PointSymbol=none]{N}{D}[s]
\pstLineAB{C}{M}
\pstLineAB[arrows=-D>]{N}{s}
\pstInterLL[PosAngle=-135]{M}{P}{D}{B}{I}
\pstProjection[]{B}{C}{I}[I']
\pstLineAB{I}{I'}
\pstSegmentMark[SegmentSymbol=MarkHashh]{D}{O}
\pstSegmentMark[SegmentSymbol=MarkHashh]{O}{P}
\pstSegmentMark[SegmentSymbol=MarkHashh]{P}{C}
\pstHomO[HomCoef=0.7,PointName=none,PointSymbol=none]{M}{I'}[f]
\pstLineAB[arrows=->]{M}{f}
\pstProjection[PosAngle=90,CodeFig=true,CodeFigColor=black]{D}{B}{I'}[G']
\pstInterLL[PosAngle=-90]{A}{B}{I'}{G'}{J}
\pstHomO[HomCoef=0.4,PointName=none,PointSymbol=none]{I'}{J}[j]
\pstLineAB[arrows=-D>]{I'}{j}
\end{pspicture}
\end{center}
  \caption{$\sqrt{3}$}\label{3}
\end{figure}

pues $ANOD,$ $AN=\frac{\sqrt{3}}{3}$ y $ON=1$ luego $\frac{ON}{AN}=\frac{1}{\frac{\sqrt{3}}{3}}=\sqrt{3},$ esto es valido para $ONMP,$ $PMBC$ pues $ON=PM=CB$ y $AN=NM=MB.$ Para verificar que $DH\perp AC$ se tiene

\subsection{El rectángulo $\sqrt{5}$}

En este rectángulo se incluye los rectángulos $\phi$ y $\sqrt{5}$  como se muestra en la figura el rectángulo $A'BCD'$ y $AB'C'D$ son rectángulos $\phi$

Se empieza construyendo un cuadrado $A'B'C'D'$ al tomemos uno de sus lados $A'B'$ divisándolo en dos segmentos iguales $A'M=MB'$ el arco generado por $MC'$ interseca a la proyección de lado $A'B'$ en los dos puntos $A$ y $B$ observe que se utilizo el mismo procedimiento para obtener el rectángulo áureo pero en este caso se obtiene dos rectángulos áureos intersecando que comparten el mismo cuadrado, observe que $A'C\perp BC'$ entonces el arco $CB$ pasa por la intersecion de esta lineas, la diagonal $AC$ pasa por la interseccion de los arcos $AB$ y $CB$

\begin{figure}
\begin{center}\psset{unit=1.2cm}
\begin{pspicture}(-2.5,-0.5)(7,4.7)%\psgrid
%\psgrid[subgriddiv=1,griddots=10]%\psframe(-0.6,-0.5)(9.2,4.7)
\rput{0}{\pstGeonode[PosAngle={-90,-90,90,90}](0,0){A'}(4,0){B'}(4,4){C'}(0,4){D'}}
\pstMiddleAB[PosAngle=-90]{A'}{B'}{M}
\pstInterLC[PosAngle=-90]{A'}{B'}{M}{C'}{A}{B}
\pstProjection{D'}{C'}{B}[C]%%%%% K
\pstProjection{D'}{C'}{A}[D]
\pspolygon[](A)(B)(C)(D)%=vlines
\pstLineAB{D}{A'}\pstLineAB{A'}{C}\pstLineAB{A}{C'}\pstLineAB{C'}{B}
\pstLineAB{A'}{D'}\pstLineAB{C'}{B'}\pstLineAB{A}{C}
\pstInterLL[PosAngle=180]{A'}{C}{B'}{C'}{O}
\pstProjection[]{B}{C}{O}[O']
\pstLineAB{O}{O'}
\pstMiddleAB[PosAngle=0]{B}{C}{N}
\pstArcOAB[linestyle=dashed,arrows=->]{N}{C}{B}
\pstInterLC[PosAngle=82,PointNameB=,PointSymbolB=none]{A}{C}{N}{B}{G}{G'}
\pstInterLL[PosAngle=90]{B}{G}{D}{C}{H}
\pstProjection[CodeFig=true,CodeFigColor=black]{B}{A}{H}[H']
\pstLineAB{B}{H}
\pstInterLL[PosAngle=185]{A'}{C}{H}{H'}{I}
\pstInterLL[PosAngle=125]{A}{C}{H}{H'}{E}
\pstProjection[PosAngle=0]{B}{C}{I}[H'']
\pstProjection[PosAngle=0]{B}{C}{E}[H''']
\pstLineAB{E}{H'''}\pstLineAB{I}{H''}
%\pcline[offset=0pt,]{->}(O)(B)
\pstInterLL[PosAngle=-55]{O'}{O}{H}{H'}{J}\pstLineAB{B}{O}
\pstInterLL[PosAngle=180]{O}{B}{H}{H'}{J'}
\pstLineAB{J'}{H''}
%\ncput*[nrot=:U]{ $\sqrt{2}$}
%
\pstArcOAB[linestyle=dashed,arrows=->]{M}{B}{A}
\pstProjection[PosAngle=-45]{B}{C}{J'}[Q]\pstLineAB{J'}{Q}
 \end{pspicture}
\end{center}
\caption{$\sqrt{5}$}\label{Uw}
\end{figure}
se obtiene $G$ intersecar el arco $BC$ con la diagonal $AC,$ $H$ al intersecar el lado $DC$ con la proyección de $BG$ finalmente $H'$ e $I$ al proyectar $H$ perpendicularmente sobre el lado $AB;$  sabe que el rectángulo $OO'CC'$ es un rectángulo $\qw$ por lo tanto $IH''CC'$ lo es, pues la diagonal $IC$ coincide con la del rectángulo $OO'CC'$ (Se demostró al principio de este capitulo que un rectángulo esta definido por el valor de la pendiente de su diagonal) se verifica que $HC=\frac{1}{5}DC$ entonces $H'BCH$ es un rectángulo $\sqrt{5},$ entonces como $H'BQJ'=1,$ y $IH''CH=\qw$  se deduce que $J'QH''I=\qw.$ También $EH'''CC'$ es un $\sqrt{5}$ pues comparten la diagonal del generador $ABCD,$ $BJ'$ es un cuadrado pues comparten la diagonal del cuadrado $B'BO'O$

\subsection{El rectángulo $\sqrt{\phi}$}

El rectángulo $\sqrt{\Phi}$ $AE'ED$ se obtiene a partir de un rectángulo áureo $\Phi$ como se muestra en la figura el rectángulo $\Phi,$ $ABCD$ se obtiene al trazar el arco $BE$  interceptando la linea $AC$ en $E,$ proyectando perpendicularmente este punto sobre la linea $AB$ se obtiene el cuarto vértice $E'$ del rectángulo $\sqrt{\phi}$
pues como se puede verificar se tiene que ${AE'}^2=\phi^2-1=2\phi+1-1=\phi\Longleftrightarrow AE'=\sqrt{\phi}$

\begin{figure}
\begin{center}
\begin{pspicture}(-0.1,-0.5)(8.3,5.5)%\psgrid[subgriddiv=1,griddots=10]
\pstGeonode[unit=5,PosAngle={-135,34,34,135},PointSymbol={*,none,none,*},PointName={default,none,none,default}]
(0,0){A}(1,0){Bb}(1,1){Cc}(0,1){D}
\pstMiddleAB[PosAngle=-90,PointName=,PointSymbol=none]{A}{Bb}{M}
\pstInterLC[PosAngle=-90,PointNameA=,PointSymbolA=none]{A}{Bb}{M}{Cc}{M''}{B}
\pstProjection[]{D}{Cc}{B}[C]
\pstInterLC[CodeFig=true,PosAngle=90,PointNameA=,PointSymbolA=none]{D}{C}{A}{B}{E''}{E}
\pstProjection[]{A}{B}{E}[E']
\pspolygon(A)(E')(E)(D)
\pspolygon(A)(B)(C)(D)
\pstArcOAB[arrows=->]{A}{B}{E}
\end{pspicture}
\end{center}

  \caption{$\sqrt{\phi}$}\label{p}
\end{figure}







\subsection{El rectángulo $\phi$}
El rectángulo de la siguiente figura tiene la única propiedad que si nosotros construimos un cuadrado sobre su lado pequeño(el menor termino del radio $\qw$), el rectángulo pequeño $aBCd$ formado a lado de este cuadrado el rectángulo original también es rectángulo $\qw,$ similar al primero. Esta operación puede ser repetido indefinidamente,ente, resultando así que los cuadrado pequeños, y pequeños y pequeños rectángulos áureos (la superficie del cuadrado y la superficie de los rectángulos formado geométricamente proverbio decreciente de radio $\dfrac{1}{\qw^2}$), como en la Figura ... Aun aun que actualmente dibujando el cuadrado , esta operación  y la proporción continua característica de la serie de los segmentos y superficies correlacionadas son subsecuentemente subconscientes  al ojo ; lo importante de esta operación  sugerente en la simple caso de una linea recta en dos segmentos  de acuerdo  a la sección áurea


\begin{figure}
\begin{center}
\psset{unit=1.2}
\begin{pspicture}(-0.6,-0.5)(7,4.7)
%\psframe(-0.6,-0.5)(7,4.7)\psgrid[subgriddiv=1,griddots=10]
\rput{0}{\pstGeonode[PosAngle={-135,-45,90,135}](0,0){O}(4,0){A}(4,4){B}(0,4){C}}
\pstMiddleAB[PosAngle=135]{O}{A}{D}
\pstInterLC[PosAngle=-90,PointSymbolA=none, PointNameA=]{O}{A}{D}{B}{P''}{P}\pstProjection{C}{B}{P}[P']%%%%%

\pspolygon[](O)(P)(P')(C)%=vlines
\pstInterLL[PosAngle=80]{O}{P'}{P}{B}{O'}\pstInterLL[PosAngle=135]{A}{B}{O}{P'}{H}
\pstProjection[PosAngle=0]{P}{P'}{H}[J]
\pstInterLL[PosAngle=-90,CodeFig=true,CodeFigColor=black]{H}{J}{P}{B}{I}
\pstProjection[CodeFig=true,CodeFigColor=black]{B}{P'}{I}[K]
\pstLineAB{A}{B}\pstLineAB{O}{P'}\pstLineAB{P}{B}\pstLineAB{H}{J}
\pstRightAngle{O}{O'}{B}
\pstSegmentMark[SegmentSymbol=MarkHashh]{O}{D}
\pstSegmentMark[SegmentSymbol=MarkHashh]{A}{D}

\pstArcOAB{B}{C}{A}\pstArcOAB{H}{A}{J}\pstArcOAB{I}{J}{K}
\pstLineAB{D}{B}
 \end{pspicture}
\end{center}
\caption{Rectangulo $\phi$}\label{Uww}
\end{figure}

wwwwwwwwwwwwwww

\section{El Pentágono y el Triángulo Áureo}
Es fácil verificar los ángulos mostrados en la figura pues los ángulos interiores de un pentágono son  como $EDC=180^\circ-\frac{360^\circ}{5}=108$ luego el ángulo $DEC=DCE=\frac{180^\circ-108^\circ}{2}=36^\circ$  ya que $I'D=I'E$ se deduce que el ángulo $DI'I''=72^\circ.$

Se prueba fácilmente que $\frac{a}{b}=\phi$ pues usando la ley de los senos en el triangulo $ADB$ se tiene que: $$\frac{a}{\sin72}=\frac{b}{\sin36}\Longleftrightarrow \dfrac{a}{b}=\frac{\sin72}{\sin36}=\frac{0.95105651629515357211\ldots}{0.5877852522924731291\ldots}=\phi.$$

Lo mismo ocurre con $\frac{DI'}{I'I}=\phi$ pues solo basta probar que los segmentos $I'I$ y $I'I''$ son iguales en efecto pues $I'I''$ es el lado del pentágono que se genera con la diagonales del pentágono $ABCDE$


\begin{figure}
\begin{center}
\begin{pspicture}(-4,-5)(5,4.8)
%\psframe(-4,-4.5)(4.5,4.5)\psgrid[subgriddiv=1,griddots=10]
\rput(0,0){\pstGeonode[CurveType=polygon,PosAngle={0,90,180,120,-90}](4,0){A}(4;72){B}(4;144){C}
(4;-144){D}(4;-72){E}}
%\pstRotation[linecolor=red, RotAngle=180, CurveType=polygon]{D}{A, B, C, D, E}
\ncline[]{-}{C}{A}
\ncline[]{-}{B}{E}
\ncline[]{-}{D}{A}
\ncline[]{-}{B}{D}
\ncline[]{-}{E}{C}
\psset{LabelRefPt=c,arrows=->,MarkAngleRadius=0.6,LabelAngleOffset=0,
LabelSep=1.3}
\pstMarkAngle[]{D}{B}{E}{$36^\circ$}
\pstMarkAngle[]{C}{E}{D}{$36^\circ$}
\pstMarkAngle[]{A}{D}{B}{$36^\circ$}
\pstMarkAngle[]{D}{C}{E}{$36^\circ$}
\pstMarkAngle[]{E}{B}{A}{$36^\circ$}

\pstInterLL[PosAngle=135]{E}{B}{A}{D}{I}% PointSymbol=square
\pstInterLL[PosAngle=55]{D}{A}{E}{C}{I'}% PointSymbol=square
\pstInterLL[PosAngle=0]{C}{E}{D}{B}{I''}% PointSymbol=square
\pstMarkAngle[]{C}{I'}{D}{$72^\circ$}
\pstMarkAngle[]{B}{I''}{C}{$72^\circ$}

\ncline[arrowscale=1]{-}{E}{C}
\ncline[offset=-14pt]{|<->|}{D}{A}
\ncput*[]{$a$}
\ncline[offset=14pt]{|<->|}{B}{A}
\ncput*[]{$b$}
 \end{pspicture}
\end{center}

\caption{El Pentágono y el Triángulo Áureo y la Relación de sus Lados}\label{R}
\end{figure}


\section{Ejemplos de Composición sobre los Rectángulos Dinámicos}
