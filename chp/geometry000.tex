
\chapter{Formas Geométricas en el Espacio}
Se coincidieran a los 5 solidos platónicos como figuras que tienen volumen y dimensiones relacionadas con el numero de oro pues como se demostrarará  las longitudes de las aristas con respecto a otros se relacionan en proporción áurea sus volúmenes se relaciona del mismo modo pero no se tratara en este libro por lo tedioso e casi inútil en el arte.

los gráficos se realizan en perspectiva por lo que no se tomara en cuenta la deducción teniendo en cuenta que el lector conoce de estos temas  para poder recrear las figuras en sus aplicaciones


Como en cada cada vertice concurren como minimo tres caras y la suma de los angulos de estas tiene que ser menor de $360^\circ$ se deduce quwe solo puede existir los isguinetes caso

\begin{itemize}
  \item 3 triangulos equilateros nos genera $3\times 60^\circ=180^\circ < 360^\circ$
  \item 4 triangulos euilateros nos genera $4\times 60^\circ=240^\circ < 360^\circ$
  \item 5 triangulos equilateros mos genera $5\times 60^\circ=300^\circ < 360^\circ$
  \item 6 triangulos equilateros nos genera $6\times 60^\circ=180^\circ < 360^\circ$ pues debe ser menor estrictamante en este caso es igual
  \item 3 cuadrados nos genera $3\times 90^\circ=270^\circ < 360^\circ$
  \item 4 cuadrados nos genera $4\times 90^\circ=360^\circ < 360^\circ$ pues debe ser menor estrictamante en este caso es igual
  \item 3 pentagonos regualares nos genera $3\times 108^\circ=324^\circ < 360^\circ$
  \item 4 pentagonos regualares nos genera $4\times 108^\circ=432^\circ < 360^\circ$ pues debe ser menor estrictamante en este caso es igual
\end{itemize}

 Es decir solo pueden existir  5 poliedros regulares o solidos platónicos

\section{El Icosaedro}
Formado por 20 caras triangulares equiláteros iguales, 12 vertices y 20 aristas
Se genera a partir de un pentágono inscrito en una circunferencia  clone este pentágono rotelo $36^\circ$ de modo que todos sus vertices coincidan con las medios arcos cuyas cuerdas son los lados del pentágono original y clonado  a partir de los vertices de este pentagon clonado y rotado  tracese líneas ortogonales al plano donde el pentágono esta, de longitudes $\frac{D}{\sqrt{5}}$, donde $D$ es el diámetro del circunferencia que inscribe a los dos pentágonos, luego une los extremos $A'',B'',B'',C'',D''$ y $E''$ para formar nuevamente un pentágono semejante alas anterior finalmente solo nos falta dos puntos.


\begin{asy}
import solids;
size(7.5cm,0);
currentprojection=perspective((45,30,50));
viewportmargin=(1mm,1mm);

draw((4,0,8)--(-4,0,8)^^(0,4,8)--(0,-4,8),dashed+darkgray);
draw("$x$",O--X,Arrow3);draw(O--3X);
draw("$y$",O--Y,Arrow3);draw(O--3Y);
draw("$z$",O--Z,Arrow3);draw(O--13Z);

path3 gene=(0,2,3)..(0,3,3.5)..(0,4,4.5)..(0,4.5,6)
..(0,4,8)..(0,1,10)..(0,2,12);
revolution vase=revolution(c=(0,0,0),gene, axis=Z, -70, 270);
draw(surface(vase),palegreen+opacity(.4));
draw(vase,m=3,frontpen=.8bp+blue,backpen=.6bp+paleblue,longitudinalpen=nullpen);
skeleton s;
vase.transverse(s,reltime(vase.g,0.5),P=currentprojection);
draw(s.transverse.back,1bp+yellow+linetype("8 8",8));
draw(s.transverse.front,1bp+yellow);

draw((0,2,3)--(0,-2,3)^^(2,0,3)--(-2,0,3),dashed+gray);
draw((0,2,12)--(0,-2,12)^^(2,0,12)--(-2,0,12),gray);

draw (gene,1bp+green);
draw ((2,0,12)..(0,2,12)..(-2,0,12)..(0,-2,12)..cycle,.4bp+red,Arrow3);

dot(Label("$a$",align=SE),(0,0,3));
dot(Label("$z$",align=SE),(0,0,8),red);
dot(Label("$b$",align=NE),(0,0,12));
draw("$r(z)$",(0,0,8)--(4,0,8),red,Arrow3);

\end{asy}

\begin{figure}
\begin{center}
\begin{asy}
import three;
size(300,0);
real radius=0.5, theta=36, phi=60;
currentprojection=perspective((0.3,1,0.7));
currentlight=Headlamp;
real r=1.1;

triple w1=radius*dir(90,0);
triple w2=radius*dir(90,360/5);
triple w3=radius*dir(90,2*(360/5));
triple w4=radius*dir(90,3*(360/5));
triple w5=radius*dir(90,4*(360/5));
triple w6=abs(w1-w2)*Z+radius*dir(90,1*(36));
triple w7=abs(w1-w2)*Z+radius*dir(90,3*(36));
triple w8=abs(w1-w2)*Z+radius*dir(90,5*(36));
triple w9=abs(w1-w2)*Z+radius*dir(90,7*(36));
triple w10=abs(w1-w2)*Z+radius*dir(90,9*(36));
triple w11=radius*dir(90,(36));
triple ww12=abs(w1-w2)*Z;
triple w12=abs(w1-w2)*Z+abs(w1-w11)*Z;
triple w=O-abs(w1-w11)*Z;
dot(Label("$A$"),w1,W);
dot(Label("$B$"),w2,dir(-90));
dot(Label("$C$"),w3,E);
dot(Label("$D$"),w4,3*N);
dot(Label("$E$"),w5,3*N);
dot(Label("$F$"),w6,dir(-135));
dot(Label("$G$"),w7,E);
dot(Label("$H$"),w8,NE);
dot(Label("$I$"),w9,NE);
dot(Label("$J$"),w10,2*W);
dot(Label("$K$"),w11,dir(-100));
dot(Label("$L$"),w12,dir(90));
dot(Label("$M$"),ww12,dir(0));
dot(Label("$N$"),w,dir(-135));
dot(Label("$O$"),O,dir(-45));
dot(w1^^w2^^w3^^w4^^w5^^w6^^w7^^w8^^w9^^w10^^w11^^w12^^w^^O^^ww12);
draw(surface(w7--w8--w12--cycle^^w10--w6--w12--cycle^^w9--w12--w10--cycle^^w9--w12--w8--cycle^^w1--w6--w10--cycle^^w5--w10--w9--cycle^^w5--w9--w4--cycle^^w9--w4--w8--cycle^^w3--w8--w7--cycle^^w1--w5--w10--cycle^^w--w5--w1--cycle^^w3--w4--w8--cycle^^w5--w4--w--cycle^^w4--w3--w--cycle^^w1--w2--w--cycle^^w2--w3--w--cycle^^w2--w6--w7--cycle^^w7--w6--w12--cycle^^w1--w2--w6--cycle^^w2--w7--w3--cycle), surfacepen=material(blue+opacity(0.5)));
draw(w1--w6--w2--w7--w3--w8--w4--w9--w5--w10--cycle^^w12--w6^^w12--w7^^w12--w8^^w12--w9^^w--w1^^w--w2^^w--w3^^w--w4^^w--w5^^w1--w2--w3--w4--w5--cycle^^w6--w7--w8--w9--w10--cycle,white);
draw(w12--w10);

draw("$36^\circ$",arc(O,w1,w11),align=3*dir(90,50),  Arrows3,light=currentlight);

path3 g1 = circle(c=O, r=radius, normal=Z);
path3 g2 = circle(c=ww12, r=radius, normal=Z);
path3 g3 = circle(c=w12, r=radius, normal=Z);
path3 g4 = circle(c=w, r=radius, normal=Z);
draw(g1^^g2^^g3^^g4, dashed);
draw(surface(g3^^g4), orange+opacity(0.5));
pen ww=linewidth(0.2mm);
draw("$r$",w11--w6,W,dashed+ww);
draw("$r$",O--w11,dashed+ww,Arrow3);
draw("$r$",O--ww12,W,dashed);
draw("$r$",w7--(w7.x,w7.y,0),W,dashed+ww);
draw("$r$",w8--(w8.x,w8.y,0),W,dashed+ww);
draw("$r$",w9--(w9.x,w9.y,0),W,dashed+ww);
draw("$r$",w10--(w10.x,w10.y,0),W,dashed+ww);
dot((w7.x,w7.y,0)^^(w8.x,w8.y,0)^^(w9.x,w9.y,0)^^(w10.x,w10.y,0));

draw("$w$",ww12--w12,W,dashed+ww);
draw("$w$",O--w,W,dashed+ww);
draw("$w$",w11--w2,N,dashed+ww);
draw("$w$",w10--(w10.x,w10.y,abs(O-w12)),W,dashed+ww,Arrows3);
draw("$w$",w3--(w3.x,w3.y,-abs(O-w)),E,dashed+ww,Arrows3);
\end{asy}
\end{center}
\caption{Icosaedro}\label{icow}
\end{figure}






\section{El Dodecaedro}


\begin{figure}
\begin{center}
\begin{asy}
import three;

size(250,0);
real radius=0.5;
currentprojection=perspective((2,9.3,2));
currentlight=Headlamp;

real r=1.1;
pen p=black;
triple w1=radius*dir(90,0);
triple w2=radius*dir(90,360/4);
triple w3=radius*dir(90,2*(360/4));
triple w4=radius*dir(90,3*(360/4));
triple w5=w1+abs(w2-w1)*Z;
triple w6=w2+abs(w2-w1)*Z;
triple w7=w3+abs(w2-w1)*Z;
triple w8=w4+abs(w2-w1)*Z;

draw(w1--w2--w3--w4--w1--w5--w6--w7--w8--w5^^w6--w2^^w7--w3^^w8--w4);
dot(Label("$E$"),w5,W);
dot(Label("$F$"),w6,SE);
dot(Label("$G$"),w7,E);
dot(Label("$H$"),w8,N);
dot(Label("$A$"),w1,S);
dot(Label("$B$"),w2,SW);
dot(Label("$C$"),w3,E);
dot(Label("$D$"),w4,S);

triple t1=(w5+w6)/2;
triple t2=(w6+w7)/2;
triple t3=(w7+w3)/2;
triple t4=(w8+w5)/2;
triple t5=(w1+w2)/2;
triple t6=(w6+w2)/2;
dot(Label("$M_1$"),t1,S);
dot(Label("$M_2$"),t2,N);
dot(Label("$M_3$"),t3,E);
dot(Label("$M_4$"),t4,N);
dot(Label("$M_5$"),t5,S);


triple w11=(w5+w7)/2;
triple w11w=t1+(w11-t1)*(1+sqrt(5))/2;
triple w12w=w11+(w11-t1)*(1-sqrt(5))/2;
triple n11=w11w+abs(w11w-w12w)*0.5*Z;
triple n12=w12w+abs(w11w-w12w)*0.5*Z;

dot(Label("$R$"),w11,N);
dot(Label("$S$"),w11w,S);
dot(Label("$T$"),w12w,NW);
dot(Label("$N_1$"),n11,N);
dot(Label("$N_2$"),n12,N);

/////////////////////////
triple w12=(w6+w3)/2;
triple w21w=t2+(w12-t2)*(1+sqrt(5))/2;
triple w22w=w12+(w12-t2)*(1-sqrt(5))/2;
triple n21=w21w+abs(w21w-w22w)*0.5*unit(w6-w5);
triple n22=w22w+abs(w21w-w22w)*0.5*unit(w6-w5);

//dot(Label("$w12$"),w12);
//dot(Label("$w21w$"),w21w);
//dot(Label("$w22w$"),w22w);
//dot(Label("$n21$"),n21);
//dot(Label("$n22$"),n22);

/////////////////////////
triple w13=(w7+w4)/2;
triple w31w=t3+(w13-t3)*(1+sqrt(5))/2;
triple w32w=w13+(w13-t3)*(1-sqrt(5))/2;
triple n31=w31w+abs(w31w-w32w)*0.5*unit(w7-w6);
triple n32=w32w+abs(w31w-w32w)*0.5*unit(w7-w6);

//dot(Label("$w13$"),w13);
//dot(Label("$w31w$"),w31w);
//dot(Label("$w32w$"),w32w);
//dot(Label("$n31$"),n31);
//dot(Label("$n32$"),n32);

/////////////////////////
triple w14=(w8+w1)/2;
triple w41w=t4+(w14-t4)*(1+sqrt(5))/2;
triple w42w=w14+(w14-t4)*(1-sqrt(5))/2;
triple n41=w41w+abs(w41w-w42w)*0.5*unit(w5-w6);
triple n42=w42w+abs(w41w-w42w)*0.5*unit(w5-w6);

//dot(Label("$w14$"),w14);
//dot(Label("$w41w$"),w41w);
//dot(Label("$w42w$"),w42w);
//dot(Label("$n41$"),n41);
//dot(Label("$n42$"),n42);


/////////////////////////
triple w15=(w5+w2)/2;
triple w51w=t6+(w15-t6)*(1+sqrt(5))/2;
triple w52w=w15+(w15-t6)*(1-sqrt(5))/2;
triple n51=w51w+abs(w51w-w52w)*0.5*unit(w6-w7);
triple n52=w52w+abs(w51w-w52w)*0.5*unit(w6-w7);

//dot(Label("$w15$"),w15);
//dot(Label("$w51w$"),w51w);
//dot(Label("$w52w$"),w52w);
//dot(Label("$n51$"),n51);
//dot(Label("$n52$"),n52);

/////////////////////////
triple w16=(w2+w4)/2;
triple w61w=t5+(w16-t5)*(1+sqrt(5))/2;
triple w62w=w16+(w16-t5)*(1-sqrt(5))/2;
triple n61=w61w+abs(w61w-w62w)*0.5*-Z;
triple n62=w62w+abs(w61w-w62w)*0.5*-Z;

//dot(Label("$w16$"),w16);
//dot(Label("$61w$"),w61w);
//dot(Label("$62w$"),w62w);
//dot(Label("$n61$"),n61);
//dot(Label("$n62$"),n62);

draw(w61w--n61, blue,Arrows3(size=5bp));
draw(w62w--n62, blue,Arrows3(size=5bp));
draw(w52w--n52, blue,Arrows3(size=5bp));
draw(w51w--n51, blue,Arrows3(size=5bp));
draw(w42w--n42, blue,Arrows3(size=5bp));
draw(w41w--n41, blue,Arrows3(size=5bp));
draw(w32w--n32, blue,Arrows3(size=5bp));
draw(w31w--n31, blue,Arrows3(size=5bp));
draw(w22w--n22, blue,Arrows3(size=5bp));
draw(w21w--n21, blue,Arrows3(size=5bp));
draw(w11w--n11, blue,Arrows3(size=5bp));
draw(w12w--n12, blue,Arrows3(size=5bp));
draw(t1-- w11w);
draw(t2--w21w);
draw(t3--w31w);
draw(t4--w41w);
draw(t6--w51w);
draw(t5--w61w);

draw(surface(
  n11--n12--w5--n42--w8--cycle^^n11--n12--w6--n22--w7--cycle
    ^^n21--n22--w6--n52--w2--cycle^^n21--n22--w7--n32--w3--cycle
    ^^n31--n32--w7--n11--w8--cycle^^n31--n32--w3--n61--w4--cycle
    ^^n41--n42--w8--n31--w4--cycle^^n41--n42--w5--n51--w1--cycle
    ^^n51--n52--w6--n12--w5--cycle^^n51--n52--w2--n62--w1--cycle
    ^^n61--n62--w2--n21--w3--cycle^^n61--n62--w1--n41--w4--cycle
  ), surfacepen=material(palegreen+opacity(0.6)));
	\end{asy}
\end{center}
\caption{\label{pointsw}El cubo y sus divisiones $\beta$ es un parámetro}
\end{figure}

Solido constituido de 12 caras pentagonales 12 aristas (8 del cubo y 12 generadas en cada una de las caras por el método que se describirá) y 30 aristas cada cada pentágono se constituye de lados que se relación con el numero de oro como  ya se vio anteriormente, veamos como  se relaciona sus diagonales  es decir las líneas que reculatan de unir puntos no contiguas

Se  obtiene un cubo las seis caras se dividen por la mitad de modo de que esas divisiones no se continúen es decir opuestos por ejemplo ($i'j'$ y $a'b'$) cada una de esas líneas  divídalos en dos segmentos iguales, sobre estas a la ves  obtenga las secciones áureas $u$ y $u'$   de los segmentos $i'o$ y $oj'$ con los segmento menores $i'o$ y $u'j'$ cercanos a las aristas del cubo respectivamente, luego de haber obtenido estos 2 secciones áureas (dos en cada unas de las caras del cubo) levántese líneas ortogonales $a'B,$ $uA$ y $iE'$ a las caras desde los puntos $u,$ $u'$ y $i'$ de longitud $ou'$ (el segmento mayor obtenido en el proceso anterior, de hallar la sección áurea) el proceso culmina al unir los vertices consecutivos del cubo con las puntos obtenidos en la proceso anterior, con los extremos de los segmentos tres ortogonales levantados anteriormente por ejemplo una de las caras del dodecaedro emerge al unir los puntos $ABeE'a$ el siguiente será el pentagono $aE'F'bE$


Ahora analicemos la longitud de los aristas, observe el plano que pasa por el centro del cubo que tiene por vertices a los puntos $ABCD$ este plano genera una sección sobre el dodecaedro llamada sección principal que es un hexágono irregular que tiene dos lados opuestos que son aristas del dodecaedro y los otro cuatro son medianas de los  de la cuatro caras


\begin{figure}
\begin{center}
\begin{pspicture}(-4.5,-4.5)(4.5,4.5)
%\psframe(-4.5,-4.5)(4.5,4.5)
\psset{SphericalCoor,Decran=20,viewpoint=10 8 4,a=1,lightsrc=80 40 50}

\psSolid[object=cube,a=3.236,action=draw*,
fillcolor=magenta!20]%
\psPoint(1,0,2.618){A}
\psPoint(-1,0,2.618){B}
\psPoint(1,0,-2.618){C}
\psPoint(-1,0,-2.618){D}
\psPoint(2.618,1,0){E}
\psPoint(2.618,-1,0){F}
\psPoint(-2.618,1,0){G}
\psPoint(-2.618,-1,0){H}
\psPoint(0,2.618,1){E'}
\psPoint(0,2.618,-1){F'}
\psPoint(0,-2.618,1){G'}
\psPoint(0,-2.618,-1){H'}

\psPoint(0,0,-1){F''}
\psPoint(0,0,1){G''}
\psPoint(0,-1,0){H''}
\psPoint(0,1,0){I''}
\psPoint(1,0,0){K''}
\psPoint(-1,0,0){J''}
  \psline[linestyle=dashed,](G'')(F'')
  \psline[linestyle=dashed,](H'')(I'')
  \psline[linestyle=dashed,](J'')(K'')

  \psline[linewidth=1.5pt](G'')(O)
  \psline[linewidth=1.5pt](o)(I'')
  \psline[linewidth=1.5pt](o)(K'')

\uput[ul](A){$\mathrm{A}$}
\uput[ur](B){$\mathrm{B}$}
\uput[d](C){$\mathrm{C}$}
\uput[ul](D){$\mathrm{D}$}

\uput[d](E){$\mathrm{E}$}
\uput[u](F){$\mathrm{F}$}
\uput[r](G){$\mathrm{G}$}
\uput[u](H){$\mathrm{H}$}

\uput[u](E'){$\mathrm{E'}$}
\uput[d](F'){$\mathrm{F'}$}
\uput[l](G'){$\mathrm{G'}$}
\uput[l](H'){$\mathrm{H'}$}
% \psline[doubleline=true]{*-*}(C1)(C0)
  \pspolygon[linewidth=1.5pt](A)(B)(D)(C)
  \pspolygon[linewidth=1.5pt](E)(F)(H)(G)
  \pspolygon[linewidth=1.5pt](E')(F')(H')(G')

\psPoint(1.618,1.618,1.618){a}
\psPoint(1.618,1.618,-1.618){b}
\psPoint(1.618,-1.618,-1.618){c}
\psPoint(1.618,-1.618,1.618){d}
\psPoint(-1.618,1.618,1.618){e}
\psPoint(-1.618,1.618,-1.618){f}
\psPoint(-1.618,-1.618,-1.618){g}
\psPoint(-1.618,-1.618,1.618){h}



\uput[ur](a){$\mathrm{a}$}
\uput[l](b){$\mathrm{b}$}
\uput[l](c){$\mathrm{c}$}
\uput[u](d){$\mathrm{d}$}
\uput[u](e){$\mathrm{e}$}
\uput[u](f){$\mathrm{f}$}
\uput[r](g){$\mathrm{g}$}
\uput[u](h){$\mathrm{h}$}


\psPoint(0,1.618,1.618){a'}
\psPoint(0,1.618,-1.618){b'}
\psPoint(1.618,1.618,0){c'}
\psPoint(1.618,-1.618,0){d'}
\psPoint(-1.618,1.618,0){e'}
\psPoint(-1.618,-1.618,0){f'}
\psPoint(0,-1.618,-1.618){g'}
\psPoint(0,-1.618,1.618){h'}

\psPoint(1.618,0,1.618){i'}
\psPoint(-1.618,0,1.618){j'}
\psPoint(1.618,0,-1.618){k'}
\psPoint(-1.618,0,-1.618){l'}


\uput[u](a'){$\mathrm{a'}$}
\uput[d](b'){$\mathrm{b'}$}

\uput[u](i'){$\mathrm{i'}$}
\uput[u](j'){$\mathrm{j'}$}


  \psline[]{*-*}(a')(b')
  \psline[linestyle=dashed]{*-*}(c')(d')
  \psline[linestyle=dashed]{*-*}(e')(f')
  \psline[linestyle=dashed]{*-*}(g')(h')
  \psline[]{*-*}(i')(j')
 \psline[linestyle=dashed]{*-*}(k')(l')

\pstInterLL[PointSymbol=o,PosAngle=40]{A}{C}{i'}{j'}{u}
\pstInterLL[PointSymbol=o,PosAngle=140]{B}{D}{i'}{j'}{u'}
\pstInterLL[PointSymbol=o,PosAngle=-40]{A}{C}{k'}{l'}{u''}
\pstInterLL[PointSymbol=o,PosAngle=-40]{B}{D}{k'}{l'}{u'''}

\pstMiddleAB[PosAngle=90]{i'}{j'}{o}
\pstMiddleAB[PosAngle=90]{F}{E}{p}
\pstMiddleAB[PosAngle=90]{H}{G}{q}

\pstInterLL[PointSymbol=o,PosAngle=-40]{G'}{E'}{a'}{b'}{i}
\pstInterLL[PointSymbol=o,PosAngle=-40]{G'}{E'}{g'}{h'}{i'}
\pstInterLL[PointSymbol=o,PosAngle=-40]{H'}{F'}{a'}{b'}{i''}
\pstInterLL[PointSymbol=o,PosAngle=-40]{H'}{F'}{g'}{h'}{i'''}

\pstInterLL[PointSymbol=o,PosAngle=-40]{G}{E}{c'}{d'}{l}
\pstInterLL[PointSymbol=o,PosAngle=-40]{G}{E}{e'}{f'}{l'}
\pstInterLL[PointSymbol=o,PosAngle=-40]{H}{F}{c'}{d'}{l''}
\pstInterLL[PointSymbol=o,PosAngle=-40]{H}{F}{e'}{f'}{l'''}


\psbrace*[linearc=0.5,linewidth=0.1pt,braceWidth=1pt,braceWidthInner=60pt, bracePos=0.30,ref=rC,rot=180](A)(C){$2\alpha\pa{\frac{\sqrt{5}+3}{2}}$}

%  \psline(a)(e)
%  \psline(b)(f)
%  \psline(d)(h)
%  \psline(c)(g)

\pspolygon[linestyle=dashed](A)(B)(e)(E')(a)%=vlines
\pspolygon[linestyle=dashed](A)(B)(h)(G')(d)%=vlines
\pspolygon[linestyle=dashed](H')(g)(D)(C)(c)%=vlines
\pspolygon[linestyle=dashed](D)(C)(b)(F')(f)%=vlines

 \psline[linestyle=dashed]{*-*}(a)(E)(b)
 \psline[linestyle=dashed]{*-*}(d)(F)(c)
 \psline[linestyle=dashed]{*-*}(e)(G)(f)
 \psline[linestyle=dashed]{*-*}(g)(H)(h)

\pspolygon[linewidth=2pt](D)(C)(p)(A)(B)(q)%=vlines


%\pstLineAB[linestyle=dashed,nodesepA=-0, nodesepB=-11, linecolor=green]{a}{d}
%\pstLineAB[linestyle=dashed,nodesepA=-0, nodesepB=-11, linecolor=green]{b}{c}
%\pstLineAB[linestyle=dashed,nodesepA=-0, nodesepB=-11, linecolor=green]{a}{e}
%\pstLineAB[linestyle=dashed,nodesepA=-0, nodesepB=-11, linecolor=green]{b}{f}
%\pstLineAB[linestyle=dashed,nodesepA=-0, nodesepB=-11, linecolor=green]{e}{h}
%\pstLineAB[linestyle=dashed,nodesepA=-0, nodesepB=-11, linecolor=green]{f}{g}

\end{pspicture}
\end{center}
\caption{\label{points}Analizando la relaciones de us dimensiones}
\end{figure}



\section{El Octaedro}


\begin{figure}
\begin{center}
\psset{unit=1.2}
\begin{pspicture}(-3,-3)(3,3)
%\psframe(-3,-3)(3,3)
\psset{viewpoint=45 30 25 rtp2xyz,Decran=120,unit=0.5}\psset{solidmemory}
\psSolid[object=plan, action=none,definition=equation, args={[0 0 1 0] 90}, name=monplan]\psset{plan=monplan}\psProjection[object=cercle,args=0 0 2,linewidth=0.5pt,linestyle=dashed,range=0 360]
\psPoint(2,0,0){A}\uput[d](A){$A$}\psdot(A)
\psPoint(0,0,0){O}\uput[r](O){$O$}\psdot(O)
\psTransformPoint[RotX=0,RotY=0,RotZ=90](2 0 0)(0,0,0){B}
\psdot(B)\uput[u](B){$B$}\psdot(B)
\psTransformPoint[RotX=0,RotY=0,RotZ=180](2 0 0)(0,0,0){C}
\psdot(C)\uput[ur](C){$C$}\psdot(C)
\psTransformPoint[RotX=0,RotY=0,RotZ=270](2 0 0)(0,0,0){D}
\psdot(D)\uput[u](D){$D$}\psdot(D)
%\pspolygon[linestyle=dashed](A)(B)(C)(D)%linestyle=dashed
\psPoint(0,0,2){F}\uput[u](F){$F$}\psdot(F)%%%%OF=LADO DEL DECAGONO
\psPoint(0,0,-2){F'}\uput[d](F'){$F'$}\psdot(F')%%%%OF=LADO DEL DECAGONO

\psline[](F)(F')%linestyle=dashed
%\axesIIID(2,2,2)(3,3,3)
\psSolid[object=octahedron,a=2,action=draw,RotZ=0](0,0,0)%

\end{pspicture}
\end{center}
 \caption{Octaedro}\label{oc}
\end{figure}

Generemos el cuadrado  $ABCD$ inscrito en una circunferencia, por el punto medio de esta circunferencia levantemos la linea $OF$ y $OF'$ de longitud $OA$ que es la mitad de la diagonal de cuadrado $ABCD,$ e fácil verificar que este sea la altura del octaedro pues cada lado es un triángulo equilátero tratemos de generar un triángulo rectángulo para poder aplicar el Teorema de Pitágoras, entonces si proyectamos el punto $O$ perpendicularmente al  segmento $AB$ obtenemos el segmento $OP$ este tiene longitud $\frac{AD}{2},$ también proyectemos el punto $F$ al segmento $AB$ así generamos el segmento $FP$ de longitud $AD\frac{\sqrt{3}}{2}$ finalmente aplicaremos el teorema de Pitágoras para obtener $OF^2+OP^2=FP^2\Longleftrightarrow OF^2=\pa{AD\frac{\sqrt{3}}{2}}^2-\pa{\frac{AD}{2}}^2
=\pa{\frac{AD}{2}}^22$ de donde $OF=\frac{AD}{\sqrt{2}}$ que verifica que $OF=AO$


Generemos el cuadrado  $ABCD$ inscrito en una circunferencia, por el punto medio de esta circunferencia levantemos la linea $OF$ y $OF'$ de longitud $OA$ que es la mitad de la diagonal de cuadrado $ABCD,$ e fácil verificar que este sea la altura del octaedro pues cada lado es un triángulo equilátero tratemos de generar un triángulo rectángulo para poder aplicar el Teorema de Pitágoras, entonces si proyectamos el punto $O$ perpendicularmente al  segmento $AB$ obtenemos el segmento $OP$ este tiene longitud $\frac{AD}{2},$ también proyectemos el punto $F$ al segmento $AB$ así generamos el segmento $FP$ de longitud $AD\frac{\sqrt{3}}{2}$ finalmente aplicaremos el teorema de Pitágoras para obtener $OF^2+OP^2=FP^2\Longleftrightarrow OF^2=\pa{AD\frac{\sqrt{3}}{2}}^2-\pa{\frac{AD}{2}}^2
=\pa{\frac{AD}{2}}^22$ de donde $OF=\frac{AD}{\sqrt{2}}$ que verifica que $OF=AO$



\section{El Exaedro o Cubo}

\begin{figure}
\begin{center}
\psset{unit=1.0}
\begin{pspicture}[showgrid=true](-4,-3)(4,3)
\psframe(-4,-3)(4,3)
\psset{lightsrc=20 40 20,viewpoint=50 60 10 rtp2xyz,Decran=100,unit=0.5,lightintensity=0.5}
\psset{solidmemory}
\psSolid[object=sphere,r=2,fillcolor=red!25,ngrid=20 25,action=draw, name=www, opacity=.3]
\psSolid[object=plan, definition=equation, args={[0 0 1 0] 90}, name=monplan,fcolor=.5 setfillopacity Yellow,]%,action=none

\psSolid[object=point,definition=solidgetsommet,args=www 0, text=B,name=B,pos=ul,]
\psset{plan=monplan}
\psProjection[object=cercle,args=0 0 2,linewidth=1pt,range=0 360]%linestyle=dashed
\psSolid[action=draw, object=cube,a=4]%


\psPoint(2,2,-2){A}\uput[d](A){$A$}\psdot(A)
\psPoint(-2,2,-2){B}\uput[r](B){$B$}\psdot(B)
\psPoint(-2,-2,-2){C}\uput[d](C){$C$}\psdot(C)
\psPoint(2,-2,-2){D}\uput[l](D){$D$}\psdot(D)
\psPoint(2,2,2){E}\uput[dl](E){$E$}\psdot(E)
\psPoint(-2,2,2){F}\uput[r](F){$F$}\psdot(F)
\psPoint(-2,-2,2){G}\uput[u](G){$G$}\psdot(G)
\psPoint(2,-2,2){H}\uput[l](H){$H$}\psdot(H)
\pspolygon[](D)(H)(F)(B)%linestyle=dashed
\pspolygon[](D)(F)(H)(B)%linestyle=dashed
\end{pspicture}
\end{center}
  \caption{Cubo}\label{cu}
\end{figure}



Formado por seis caras cuadrados iguales, ocho vertices y doce aristas la sección principal pasa por dos aristas opuestas y hay seis des estas secciones en un cubo tales como, debemos destacar que $\frac{DF}{3}$
Hay que demostrar como se forma el cubo y cual es la proporción entre su lado y el diámetro de la esfera que lo circumscribe exactamente, tómese el diámetro de la esfera en la que se prepone colocarlo exactamente y sea este la linea $AB,$ sobre la cula se trasa el semicírculo $ADB$ luego divídalas el diámetro en el punto $C$ de manera que $AC=2BC$ tracsece la linea $CD$ perpendicular a la linea $AB$ además tracese las líneas $BC$ y $CA.$ Haga luego un cuadrado cuyos lados iguales a la linea $BD$

luego se verifica que $3BD^2=AB^2\Longleftrightarrow AB=\sqrt{3}BD$


\section{El Tetraedro}
\begin{figure}
\begin{center}
 \begin{pspicture}[showgrid=true](-5,-2.5)(5,4)
 \psset{lightsrc=50 30 20,viewpoint=20 20 5,Decran=50}
 \psset{solidmemory,lightintensity=0.5,visibility=false}
  \psSolid[object=tetrahedron,r=2.1,fillcolor=orange!50,name=sed,action=draw,opacity=0.5, show=all]
 	\psSolid[object=point,definition=solidgetsommet,args=sed 1, text=A,name=A,pos=dr]
  \psSolid[object=point,definition=solidgetsommet,args=sed 0, text=B,name=B,pos=ul,]
  \psSolid[object=vecteur,args=0 0 2,name=v,action=none](B)
  \psSolid[object=point,definition=orthoprojplane3d,args=B A v, text=O,name=O,pos=ul]
  \psSolid[object=plan,definition=solidface,args=sed 3, showbase,fillcolor=red,action=draw,name=t,base=-2 2 -2 2]
	\psSolid[object=line,args=O B]
	\end{pspicture}
\end{center}
  \caption{opcion $\int_5sed$}\label{u}
\end{figure}

El tetraedro es mu y fácil de construir sea el triángulo equilátero $ABC$ a partir de su centro $O$ se lavanda una  ortogonal $OF=r\sqrt{2}$ donde $r$ es la radio de la circunferencia que circumscribe al triángulo
