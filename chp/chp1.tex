\chapter{Curvas}
\pagenumbering{arabic}
\setcounter{page}{1}






En este capítulo se observara la definición y las características de las curvas (lineas).


\begin{defn}[Curva]
Es una coleccion de puntos en el espacio. En matemática (inicialmente estudiado en geometría elemental y, de forma más rigurosa, en geometría diferencial), la curva (o línea curva) es una línea continua de una dimensión, que varía de dirección paulatinamente. \cite{hilbert2020geometry}. ``\usebibentry{hilbert2020geometry}{title}''
\end{defn}

Ejemplos sencillos de curvas cerradas simples son la elipse o la circunferencia o el óvalo, el cicloide; ejemplos de curvas abiertas, la parábola, la hipérbola y la catenaria y una infinidad de curvas estudiadas en la \textbf{geometría analítica plana}. La recta asume el caso límite de una circunferencia de radio de curvatura infinito y de curvatura 0; además, una recta es la imagen homeomorfa de un intervalo abierto. Todas las curvas tienen dimensión topológica igual a 1. La noción curva, conjuntamente con la de superficie, es uno de los objetos primordiales de la geometría diferencial, ciertamente con profusa aplicación de las herramientas del cálculo diferencial

\section{Tangente en un punto de una curva}
\begin{defn}[Tangente] In the definition of defn you need to use the first optional argument of  newtheorem to indicate that \cite{hilbert2020geometry}. ``\usebibentry{hilbert2020geometry}{title}''  this environment shares the counter of the previously defined thm environment.\end{defn}
\cite{reyes} \index{wwwww}\cite{www}


\begin{figure}[!ht]
	\centering
	\begin{asy}
	import graph3;
	import three;
	size3(200,0);
	currentprojection=perspective(4,6,3);
	real x(real t) {return 1+cos(2pi*t);}
	real y(real t) {return 1+sin(2pi*t);}
	real z(real t) {return t;}
	real x1(real t) {return -2pi*sin(2pi*t);}
	real y1(real t) {return 2pi*cos(2pi*t);}
	real z1(real t) {return 1;}
	real x2(real t) {return -4pi^2*cos(2pi*t);}
	real y2(real t) {return -4pi^2*sin(2pi*t);}
	real z2(real t) {return 0;}
	real t=0.7;
	triple pp=(x(t),y(t),z(t));
	triple tt=(x1(t),y1(t),z1(t));
	triple nn=(x2(t),y2(t),z2(t));
	triple ww=cross(tt,nn);
	triple cc=pp+length(ww)/length(tt)^3*unit(nn);
	path3 p=graph(x,y,z,0,1,operator --);
	dot((x(t),y(t),z(t)));
	draw(pp--pp+unit(tt), Arrow3, p=gray(0.6), light=currentlight );
	draw(pp--pp+unit(ww), Arrow3, p=orange, light=currentlight );
	draw(pp--pp+length(ww)/length(tt)^3*unit(nn),Arrow3, p=orange, light=currentlight );
	//draw(pp--pp+unit(nn), orange, Arrow3 );
	dot(pp--pp+length(ww)/length(tt)^3*unit(nn), orange );
	draw((plane(O=pp+unit(tt)+unit(nn), -2*unit(tt), -2*unit(nn))), gray + 0.1cyan);
	//draw(surface(pp+unit(tt)-unit(nn)--pp+unit(tt)+unit(nn)--pp-unit(tt)+unit(nn)--pp-unit(tt)-unit(nn)--cycle),orange);
	draw(surface(plane(O=pp+unit(ww)+unit(nn), -2*unit(ww), -2*unit(nn))),blue+opacity(.2));
	draw(surface(plane(O=pp+unit(tt)+unit(ww), -2*unit(tt), -2*unit(ww))),magenta+opacity(.5));
	draw(p, Arrow3);
	//path3 g=pp..cc+unit(tt)..cc+unit(nn)..cc-unit(tt)..cycle;
	//draw(g);
	path3 g = circle(c=cc, r=length(ww)/length(tt)^3, normal=ww);
	draw(g);
	draw(surface(g), blue+opacity(0.3));
	axes3("$x$","$y$","$z$", Arrow3);
	\end{asy}
	\caption{Curva 3D con las rectas: tangente normal y binormal (Triedro de Frenet-Serret) además los planos: osculador, rectificante y normal}
\end{figure}

Curva 3D con las rectas: tangente normal y binormal además los planos: osculador, rectificante y normal


\begin{prop}[wwwwwwwwwwwwwwwwwwwwwwww] In the definition of defn you need to use the first optional argument of  newtheorem to indicate that this environment shares the counter of the previously defined thm environment.\end{prop}



\begin{defn}[wwwwwwwwwwwwwwwwwwwwwwww] In the definition of defn you need to use the first optional argument of  newtheorem to indicate that \cite{hilbert2020geometry}. ``\usebibentry{hilbert2020geometry}{title}''  this environment shares the counter of the previously defined thm environment.\end{defn}
	\cite{reyes} \index{wwwww}  \cite{www}   \usebibentry{reyes}{title}


\begin{figure}[!ht]
\centering
\begin{asy}
size(300,0);
import markers;
import geometry;
import math;

pair A=0, B=(1,0), C=(0.7,1), D=(-0.5,0), F=rotate(-90)*(C-B)/2+B;

draw(A--B);
draw(A--C);
pen p=linewidth(1mm);
draw(B--C,p);
draw(A--D);
draw(B--F,p);
label("$A$",A,SW);
label("$B$",B,S);
label("$C$",C,N);
dot(Label("$D$",D,S));
dot(Label("$F$",F,N+NW));
markangle(A,C,B);
markangle(scale(1.5)*"$\theta$",radius=40,C,B,A,ArcArrow(2mm),1mm+red);
markangle(scale(1.5)*"$-\theta$",radius=-70,A,B,C,ArcArrow,green);
markangle(Label("$\gamma$",Relative(0.25)),n=2,radius=-30,A,C,B,p=0.7blue+2);
markangle(n=3,B,A,C,marker(markinterval(stickframe(n=2),true)));
pen RedPen=0.7red+1bp;
markangle(C,A,D,RedPen,marker(markinterval(2,stickframe(3,4mm,RedPen),true)));
drawline(A,A+dir(A--D,A--C),dotted);
perpendicular(B,NE,F-B,size=10mm,1mm+red,
TrueMargin(linewidth(p)/2,linewidth(p)/2),Fill(yellow));
\end{asy}
\caption{geometry}
\end{figure}




If the optional boolean argument check is false, no check will be made that the file exists. If the file does not exist or is not readable, the function bool error(file) will return true. The first character of the string comment specifies a comment character. If this character is encountered in a data file, the remainder of the line is ignored. When reading strings, a comment character followed immediately by another comment character is treated as a single literal comment character. If Asymptote is compiled with support for libcurl, name can be a URL.

Unless the -noglobalread command-line option is specified, one can change the current working directory for read operations to the contents of the string s with the function string cd(string s), which returns the new working directory. If string s is empty, the path is reset to the value it had at program startup.

When reading pairs, the enclosing parenthesis are optional. Strings are also read by assignment, by reading characters up to but not including a newline. In addition, Asymptote provides the function string getc(file) to read the next character (treating the comment character as an ordinary character) and return it as a string.




\section{La Sección Áurea}

Sea el segmento $AB$ dividamoslo de la siguiente manera, tomemos $\frac{AB}{2}$ coloquemos este segmento de manera que sea perpendicular a $AB$ en cualquiera de los extremos en este caso sea $B$ interceptemos la linea $AC$ con el arco $BD$ centrado en $C$ la cual nos da el punto $D$ a partir de este punto tracemos el arco $DE$ centrado en $A$ hallando de este modo el punto $E$ que divide al segmento $AB$ en \textsc{extrema y media razón} o \textsc{sección áurea} \cite{Phillips} y \cite{variablei}.

\begin{figure}[!ht]
	\begin{center}
	\end{center}
	\caption{Sección áurea de un segmento}
	\label{Hw}
\end{figure}

Es decir podemos empezar diciendo que $\frac{AB}{AE}=\frac{AE}{EB}=1.61833....=\phi $ es el numero de oro \cite{surhone2010shapiro}  \cite{jackson2012research}. ``\usebibentry{jackson2012research}{title}''

Deduzcamos y averigüemos de donde nace el \texttt{número de oro}; empecemos con la frase celebre que dice mucho, lo genera y esta relacionado con este  número: \textbf{\textit{El todo sobre la parte mayor y la parte mayor sobre la menor}} \cite{Heinz}. \cite{hilbert2020geometry}. ``\usebibentry{hilbert2020geometry}{title}''

\subsection{Análisis de la sección áurea}
Tomando la figura \ref{ec1} y recordando que si se tiene la ecuaion $$ax^2+bx+c=0$$ la raices que satisfacen esta ecuacion son $x=\frac{-b\pm\sqrt{b^2-4\pa{a}\pa{c}}}{2a}$


\begin{figure}[!ht]
	\begin{center}

	\end{center}
	\caption{La Parabola $x^2-kx-k^2=y$ y los puntos $S$ y $S'$}\label{ec1}
\end{figure}


Cuando todo el segmento permanece constante y el segmento menor es \index{constante} constante para simplificar consideremos esa constante igual 1 luego según la figura  se tiene que $\theta=1$
 wwwwwwwwwwwwwwwwwwwwwwwwwwww
\begin{longtable}{ccc>{\color{blue}}c>{\color{blue}}c}
	\caption{Combinaciones de los tres segmentos de la seccion aurea.}
	\label{tab:w1wwwww}\\
	\toprule
	\textbf{Ecuación} & \textbf{Simplicación} & \textbf{Raices}& \multicolumn{2}{c}{\textbf{Raices simplicación}}\\\midrule
	 &  &  & $x_1$ & $x_2$ \\
	\midrule
	\endfirsthead % <-- This denotes the end of the header, which will be shown on the first page only
 \multicolumn{4}{c}{{\bfseries \tablename\ \thetable{} -- continua de la página anterior}} \\
	\toprule
	\textbf{Value 1} & \textbf{Value 2} & \textbf{Value 3}& \multicolumn{2}{c}{\textbf{Raices}}\\\midrule
	$\alpha$ & $\beta$ & $\gamma$ & $x_1$ \\
	\midrule
	\endhead
	\multicolumn{4}{c}{{Continúa en la proxima página}} \\ \midrule
	\endfoot
	\bottomrule
	\endlastfoot
	$\frac{x}{x-1}=\frac{x-1}{1}$&$ x^2-3x+1=0 $ & $x=\frac{3\pm\sqrt{5}}{2}$  & $x_1=\fpeval{round((1+sqrt(5))/2,3)}$ & $x_2=\fpeval{round((1-sqrt(5))/2,3)}$\\\midrule
	$\frac{x+1}{x}=\frac{x}{1}$&$ x^2-x-1=0$     & $x=\frac{1\pm\sqrt{5}}{2}$  & $x_1=\fpeval{round((-1+sqrt(5))/2,3)}$ & $x_2=\fpeval{round((1-sqrt(5))/2,3)}$\\\midrule
	$\frac{x+1}{x}=\frac{x}{1}$&$ x^2-x-1=0$     & $x=\frac{1\pm\sqrt{5}}{2}$  & $x_1=\fpeval{round((3+sqrt(5))/2,3)}$ & $x_2=\fpeval{round((1-sqrt(5))/2,3)}$\\\midrule
	$\frac{1}{x}  =\frac{x}{1-x}$&$ x^2+x-1=0 $  &  $x=\frac{-1\pm\sqrt{5}}{2}$& $x_1=\fpeval{round((-3+sqrt(5))/2,3)}$ & $x_2=\fpeval{round((1-sqrt(5))/2,3)}$\\\midrule
	$\frac{x+1}{x}=\frac{x}{1}$&$ x^2-x-1=0$     & $x=\frac{1\pm\sqrt{5}}{2}$  & $x_1=\fpeval{round((1+sqrt(5))/2,3)}$ & $x_2=\fpeval{round((1-sqrt(5))/2,3)}$\\\midrule
	$\frac{x+1}{x}=\frac{x}{1}$&$ x^2-x-1=0$     & $x=\frac{1\pm\sqrt{5}}{2}$  & $x_1=\fpeval{round((1+sqrt(5))/2,3)}$ & $x_2=\fpeval{round((1-sqrt(5))/2,3)}$\\
\end{longtable}


Las Ecuaciones coincide dos a dos; si se reemplaza cada una des us raíces sobre sus correspondientes  \index{ecuaciones} se obtiene $\frac{1\pm\sqrt{5}}{2}$ en efecto solamente tenemos tres ecuaciones ya que ellos coinciden

\begin{longtable}{ccccc}
	\caption{Combinaciones de los tres segmentos de la seccion aurea.}
	\label{tab:w1wwwww}\\
	\toprule
	\textbf{Ecuación} & \textbf{Simplicación} & \textbf{Raices}& \multicolumn{2}{c}{\textbf{Raices simplicación}}\\\midrule
	 &  &  & $x_1$ & $x_2$ \\
	\midrule
	\endfirsthead % <-- This denotes the end of the header, which will be shown on the first page only
 \multicolumn{4}{c}{{\bfseries \tablename\ \thetable{} -- continua de la página anterior}} \\
	\toprule
	\textbf{Value 1} & \textbf{Value 2} & \textbf{Value 3}& \multicolumn{2}{c}{\textbf{Raices}}\\\midrule
	$\alpha$ & $\beta$ & $\gamma$ & $x_1$ \\
	\midrule
	\endhead
	\multicolumn{4}{c}{{Continúa en la proxima página}} \\ \midrule
	\endfoot
	\bottomrule
	\endlastfoot
	$\frac{x}{x-1}=\frac{x-1}{1}$&$ x^2-3x+1=0 $ & $x=\frac{3\pm\sqrt{5}}{2}$  & $x_1=\fpeval{round((3+sqrt(5))/2,3)}$ & $x_2=\fpeval{round((3-sqrt(5))/2,3)}$\\\midrule
	$\frac{x+1}{x}=\frac{x}{1}$&$ x^2-x-1=0$     & $x=\frac{-1\pm\sqrt{5}}{2}$  & $x_1=\fpeval{round((-1+sqrt(5))/2,3)}$ & $x_2=\fpeval{round((-1-sqrt(5))/2,3)}$\\\midrule
	$\frac{x+1}{x}=\frac{x}{1}$&$ x^2-x-1=0$     & $x=\frac{1\pm\sqrt{5}}{2}$  & $x_1=\fpeval{round((1+sqrt(5))/2,3)}$ & $x_2=\fpeval{round((1-sqrt(5))/2,3)}$\\
\end{longtable}


\subsection{Propiedades del numero $\phi$}

La sección áurea, la proporción geométrica definidas en el capitulo precedente, $\frac{1+\sqrt{5}}{2}=1.618...$ la raíz positiva de la ecuación $x^2=x+1,$ tiene una cierto número de propiedades  algebraicas  y geométricas   donde podemos hacer en los remarkable la propiedad algebraica  en alguna manera  com $\pi$ (el radio entre alguna circunferencia y su diámetro) y $e=\lim_{n\longrightarrow \infty}\pa{1+\frac{1}{n}}^n$ donde $n\in \mathbb{N}$ son los numeros trascendentes  mas rescatables  .
Si  se sigue nosotros llamamos este numero, radio, o proporción $\qw,$  y tenemos las siguientes propiedades interesantes:

$$\phi=\frac{2+\sqrt{5}}{2}=1.61803398875...$$
(así que $1.618\ldots $ es una aproximación  muy cercana)

$\qw^2=2.618...=\frac{\sqrt{5}+3}{2}$ y
$\frac{1}{\qw}=0.618...=\frac{\sqrt{5}-1}{2}$


La sección áurea, la proporción geométrica definidas en el capitulo precedente, $\frac{1+\sqrt{5}}{2}=1.618...$ la raíz positiva de la ecuación $x^2=x+1,$ tiene una cierto número de propiedades  algebraicas  y geométricas   donde podemos hacer en los remarkable la propiedad algebraica  en alguna manera  com $\pi$ (el radio entre alguna circunferencia y su diámetro) y $e=\lim_{n\longrightarrow \infty}\pa{1+\frac{1}{n}}^n$ donde $n\in \mathbb{N}$ son los numeros trascendentes  mas rescatables  .
Si  se sigue nosotros llamamos este numero, radio, o proporción $\qw,$  y tenemos las siguientes propiedades interesantes:

\begin{longtable}{lllc}
	\caption{Convergencia de la sucesión de Fibonacci al número áureo }
	\label{tab:w1wwwww}\\
	\toprule
	\textbf{N} & \textbf{$F_n$} & \textbf{$\frac{F_n}{F_{n-1}}$}&\textbf{$\phi-\frac{F_n}{f_{n-1}}$}\\\midrule
	 %w& w & w  \\
	%\midrule
	\endfirsthead % <-- This denotes the end of the header, which will be shown on the first page only
 \multicolumn{4}{c}{{\bfseries \tablename\ \thetable{} -- continua de la página anterior}} \\
	\toprule
	\textbf{N} & \textbf{$F_n$} & \textbf{$\frac{F_n}{F_{n-1}}$}&\textbf{$\phi-\frac{F_n}{f_{n-1}}$}\\\midrule
	 %w& w & w  \\
	%\midrule
	\endhead
	\midrule\multicolumn{4}{c}{{Continúa en la proxima página}} \\\midrule
	\endfoot
	\bottomrule
	\endlastfoot
1	&	1	&		&		\\
2	&	1	&	1	&	0.618033988749895	\\
3	&	2	&	2	&	-0.381966011250105	\\
4	&	3	&	1.5	&	0.118033988749895	\\
5	&	5	&	1.66666666666667	&	-0.0486326779167718	\\
6	&	8	&	1.6	&	0.0180339887498948	\\
7	&	13	&	1.625	&	-0.0069660112501051	\\
8	&	21	&	1.61538461538462	&	0.00264937336527948	\\
9	&	34	&	1.61904761904762	&	-0.00101363029772417	\\
10	&	55	&	1.61764705882353	&	0.000386929926365465	\\
11	&	89	&	1.61818181818182	&	-0.000147829431923263	\\
12	&	144	&	1.61797752808989	&	5.6460660007307E-05	\\
13	&	233	&	1.61805555555556	&	-2.15668056606777E-05	\\
14	&	377	&	1.61802575107296	&	8.23767693347577E-06	\\
15	&	610	&	1.61803713527851	&	-3.14652861965747E-06	\\
16	&	987	&	1.61803278688525	&	1.20186464891425E-06	\\
17	&	1597	&	1.61803444782168	&	-4.59071787028975E-07	\\
18	&	2584	&	1.61803381340013	&	1.75349769593325E-07	\\
19	&	4181	&	1.61803405572755	&	-6.69776591966098E-08	\\
20	&	6765	&	1.61803396316671	&	2.55831884565794E-08	\\
21	&	10946	&	1.6180339985218	&	-9.77190839357434E-09	\\
22	&	17711	&	1.61803398501736	&	3.73253694618825E-09	\\
23	&	28657	&	1.6180339901756	&	-1.4257022229458E-09	\\
24	&	46368	&	1.61803398820533	&	5.44569944693762E-10	\\
25	&	75025	&	1.6180339889579	&	-2.08007167046276E-10	\\
26	&	121393	&	1.61803398867044	&	7.94517784896698E-11	\\
27	&	196418	&	1.61803398878024	&	-3.03477243335237E-11	\\
28	&	317811	&	1.6180339887383	&	1.15918386001113E-11	\\
29	&	514229	&	1.61803398875432	&	-4.42756942220512E-12	\\
30	&	832040	&	1.6180339887482	&	1.69131375571396E-12	\\
31	&	1346269	&	1.61803398875054	&	-6.45927755726916E-13	\\
32	&	2178309	&	1.61803398874965	&	2.4669155607171E-13	\\
33	&	3524578	&	1.61803398874999	&	-9.41469124882133E-14	\\
34	&	5702887	&	1.61803398874986	&	3.59712259978551E-14	\\
35	&	9227465	&	1.61803398874991	&	-1.37667655053519E-14	\\
36	&	14930352	&	1.61803398874989	&	0	\\
37	&	24157817	&	1.6180339887499	&	0	\\
\end{longtable}

\begin{itemize}
	\item  Se sabe que la ecuacion $\qw^2-\qw-1=0$ se satisface luego podemos operar de infinitas maneras  esta ecuación trasmutando, dividiendo y multiplicando términos \begin{align*}
		\qw^2&=\qw+1=\qw+1\\
		\qw^3&=\qw^2+\qw=\qw+1+\qw=2\qw+1\\
		\qw^4&=\qw^3+\qw^2=2\qw+1+\qw+1=3\qw+2\\
		\ldots &=\ldots\ldots\\
		\qw^n&=\qw^{n-1}+\qw^{n-2}==i\qw+j\\
		\qw^{n+1}&=\qw^{n}+\qw^{n-1}=m\qw+n\\
		\qw^{n+2}&=\qw^{n+1}+\qw^{n}=\pa{i+m}\qw+\pa{j+n}
	\end{align*}
	Esto también es valido para exponentes negativos $\qw=1+\frac{1}{\qw}=\qw^0+\qw^{-1},$  luego

	\item Las series \begin{align*}
		\frac{1}{\qw^{2}}=\qw^{-2}&=\qw^{-3}+\qw^{-4}=\frac{1}{\qw^{3}}+\frac{1}{\qw^{4}}\\
		\frac{1}{\qw^{3}}=\qw^{-3}&=\qw^{-4}+\qw^{-5}=\frac{1}{\qw^{4}}+\frac{1}{\qw^{5}}\\
		\ldots&=\ldots\\
		\frac{1}{\qw^{n}}=\qw^{-n}&=\qw^{-\pa{n+1}}+\qw^{-\pa{n+2}}=\frac{1}{\qw^{\pa{n+1}}}+\frac{1}{\qw^{\pa{n+2}}}\\
	\end{align*}

	\item $2=\qw+\frac{1}{\qw^2}$ pues de $\qw^{3}=\qw^{2}+\qw=\pa{\qw+1}+\qw=2\qw+1$ porque $\qw^{2}=\qw+1$ luego $\qw^{3}=2\qw+1\Longleftrightarrow 2=\qw^2-\frac{1}{\qw}=\qw+1-\frac{1}{\qw}=\qw+\frac{\qw\pa{\qw-1}}{\qw^2}=\qw+\frac{1}{\qw^2}$

	\item $\qw=\frac{1}{\qw-1}$ en efecto de $\qw^2-\qw-1=0$ al factorizar $\qw$ y despejar 1 se obtiene $\phi\pa{\phi-1}=1$ (recuerde que $\qw\neq 0\Longrightarrow \qw-1\neq 0$) ambos miembros de la igualdad y despejar $\qw$ es decir $\qw=\frac{1}{\qw-1}$


	\item La sucesión $$\qw=1+\cfrac{1}{1+\cfrac{1}{1+\cfrac{1}{1+\cfrac{1}{1+\ldots}}}}$$
	Pues $\qw={\qw}^0+{\qw}^{-1}=1+\frac{1}{\qw}$ por la ecuación obtenida anteriormente, es decir al reemplazar $\qw=1+\frac{1}{\qw}$ en el denominador del lado derecho de ésta ecuación se obtiene $\qw=1+\frac{1}{1+\frac{1}{\qw}}$ luego al iterar llegamos al resultado deseado
\end{itemize}


\iffalse

\STautoround{9}
\nprounddigits{15}
\let\PC\%
\newcommand\Mystrut{\rule[-1.5ex]{0pt}{0.5ex}}
\begin{spreadtab}{{tabularx}{0.5\linewidth}{c c<\PC}}
\toprule
\multicolumn{2}{c}{Convergence at $x=\color{red}:={0.5}$}\\[1.5ex]
@$n$ & e^a1\SThidecol & @ $\displaystyle e^{\numprint{<<a1>>}}-\sum_{k=0}^n\frac{\numprint{<<a1>>}^k}{k!}$\\[3ex]\midrule
$\color{red}:={1}$& a1^[-1,0]/fact([-1,0]) & \STcopy{v}{b!2-b3}\\
$\phi^\STcopy{v}{a3+1}$& \STcopy{v}{a!1^a4/fact(a4)+b3}& \\
$\phi^:={}$& & \\
$\phi^:={}$& & \\
$\phi^:={}$&\color{red}:={} & \\
$\phi^:={}$& & \\
$\phi^:={}$& & \\
$\color{red}\phi^:={}$& & \\
$\phi^:={}$& & \\
$\phi^:={}$& & \\
$\phi^:={}$& & \\
$\phi^:={}$& & \\
$\phi^:={}$& & \\
$\phi^:={}$& & \\
$\phi^:={}$& & \\
$\phi^:={}$& & \\
$\phi^:={}$& & \\
$\phi^:={}$& & \\
$\phi^:={}$& & \\
$\phi^:={}$& & \\
$\phi^:={}$& & \\
$\phi^:={}$& & \\
$\phi^:={}$& & \\
$\phi^:={}$& & \\
$\phi^:={}$& & \\
$\phi^:={}$& & \\
$\phi^:={}$& & \\
$\phi^:={}$& & \\
$\phi^:={}$& & \\
$\phi^:={}$& & \\
\end{spreadtab}

\begin{table}
  \caption{Sucecion de Fibonacci}
  \vspace{0.5cm}
  % \label{}
\centering
  \begin{spreadtab}{{tabularx}{0.8\linewidth}{c >\Mystrut>{\color{orange}}c N{1}{15} N{2}{15}}}
  \toprule
  @$n$ & @$F_n$ & @\color{blue}\hfill{$\dfrac{F_n}{F_{n-1}}$}\hfill\null& @ \color{yellow}\hfill{$\varphi-\dfrac{F_n}{F_{n-1}}$}\hfill\null\\[2ex]\midrule
  \color{orange}:=1    & \color{magenta}:=1    &                      & \\
  \STcopy{v}{a2+1} & \color{red}:=1 & \STcopy{v}{b3/b2} & (1+5^0.5)/2-[-1,0] \\
                    & \STcopy{v}{b2+b3} &                      & \STcopy{v}{d!3+1-c4} \\
  & & & \\
  & & & \\
  & & & \\
  & & & \\
  & & & \\
  & & & \\
  & & & \\
  & & & \\
 &&  & \\
  & & & \\
  & & & \\
  & & & \\
  & & & \\
   & & & \\
  \bottomrule
  \end{spreadtab}

\end{table}
\fi
la progresión geométrica  de la serie $$1,\qw,\qw^2,\qw^3,\ldots,\qw^n,\ldots$$ cada termino es la suma de los numeros anteriores; esta promediad viene al mismo tiempo aditivo y geométrico es característica de esta serie y es una razón para su rol en la evolución de los organismos, especialmente en la botánica.
en la progresión diminuta  $$1,\frac{1}{\qw},\frac{1}{\qw^2},\frac{1}{\qw^3},\ldots,\frac{1}{\qw^m}$$ tenemos  $\frac{1}{\qw^m}=\frac{1}{\qw^{m+1}}-\frac{1}{\qw^{m+2}}$ (cada termino es la suma de los dos siguientes  unos) y $$\qw=\frac{1}{\qw}+\frac{1}{\qw}+\frac{1}{\qw}+\ldots+\frac{1}{\qw}+\ldots$$ donde $m$ se genera indefinidamente.
La construcción rigurosa  del radio o proporción de $\qw$ es muy simple, porque de su valor $\frac{1+\sqrt{5}}{2}.$ La Figura~\ref{KK} muestra como, empezando de un segmento mayor  $AB,$ para construir el segmento  menor $BC$ tal que $\frac{AB}{BC}=\qw,$ y como inversamente, empezando de un segmento completo $AC,$ para colocar el punto  $B$ dividiendo  su en el dos segmentos $AB$ y $BC$ relativo por la sección áurea  (otro construcción en la figura 3). Este mas lógico asimétricas division de una linea, o de un superficie, es también el mas satisfactorio para los ojos; este tiene un significado







El principio aplica siempre e un de un designio la presencia de una proporción característica  de un cadena  de un proporción relacionada (esto es una noción impropio  donde sera ilustrado después) produce la recurrencia de forma similar, pesero la sugestión subconsciente mencionada anteriormente especialmente asociada con la Sección Áurea porque de la propiedad de algún a progresión geométrica de radio $\qw$ o $\frac{1}{\qw}$ es decir $$a,a\qw,a\qw^2,a\qw^3,\ldots,a\qw^n,\ldots$$  ó $$a, \frac{a}{\qw},\frac{a}{\qw^2},\frac{a}{\qw^3},\ldots,\frac{a}{\qw^n},\ldots$$



\begin{figure}[!ht]
	\begin{center}
	\end{center}
	\caption{Construcción del segmento menor $BC$ a partir del segmento mayor $AB$}\label{KK}
\end{figure}

Concentrándonos en el triángulo  $A'BA,$   al rotar esta figura obtenemos la siguiente  y  se observa que $AA'=\frac{AB}{2}$ este método de obtener la sección áurea se vio al principio es decir el punto $Y$ es la sección áurea con respecto a la linea $AB$ como lo es el punto $A$ con respecto a la linea $OS$

\begin{figure}[!ht]
	\begin{center}

	\end{center}
	\caption{Construcción del segmento menor $BY$ a partir del segmento mayor $AB,$ $AY=UB$; $\frac{OA}{AS}=\frac{OS}{OA}=\frac{AU}{UB}=\frac{\sqrt{5}+1}{2}$}\label{H}
\end{figure}


\emph{}


\begin{figure}[!ht]
	\begin{center}
	\end{center}
	\caption{$\frac{AB}{YB}=\frac{A''R'}{R'O}=\phi.$ Se unió los procedimientos anteriores}\label{j}
\end{figure}

En la figura \ref{sed} se prueba que $\frac{AB}{BC}=\frac{BC}{CD}=\frac{AC}{AB}=\phi$


\begin{figure}[!ht]
	\begin{center}

	\end{center}
	\caption{$\frac{AB}{BC}=\frac{BC}{CD}=\frac{AC}{AB}=\phi$}\label{sed}
\end{figure}



\section{Rectángulos dinámicos estructurales}


Los rectángulos dinámicos se caracterizan por  tener proporciones no racionales es decir irracionales en la Figura \ref{dynamics} observamos que los rectángulo $\sqrt{2},\sqrt{3}, \sqrt{5},$... son dinámicos, excepto el $\sqrt{4}=2$ que es un número racional también se observa que a partir de un cuadrado Figura~\ref{dynamics} se pueden construir sucesivamente estos rectángulo en algunos  casos obviamente mediante este proceso se podrán hallar rectángulos  no dinámicos.

La principal aplicación esta siempre en el diseño y la presencia en el arte plástico es una característica proporcionado por la geometría derivada de la sección áurea o de una cadena de proporciones relacionadas (este es una noción importante donde será ilustrado después), donde se produce la recurrencia de formas similares, pero la sugestión mencionad arriba es especialmente asociada con la Sección Áurea porque ella posee propiedades muy interesantes con la infinita variedad de progresión geométrica de radio
La principal aplicación esta siempre en el diseño y la presencia en el arte plástico es una característica proporcionado por la geometría derivada de la sección áurea o de una cadena de proporciones relacionadas (este es una noción importante donde será ilustrado después), donde se produce la recurrencia de formas similares, pero la sugestión mencionad arriba es especialmente asociada con la Sección Áurea porque ella posee propiedades muy interesantes con la infinita variedad de progresión geométrica de radio $\phi$


\begin{figure}[!ht]
	\begin{center}

	\end{center}
	\caption{Rectángulos Dinámicos $\sqrt{2},$ $\phi,$ $\sqrt{3},$ $\sqrt{5},$ ...}
	\label{dynamics}
\end{figure}

Como el rectángulo $ARKC$ denotado por $\sqrt{2}$, $ASLC$ denotado por $\sqrt{3}$, $OTMC$ denotado por $\sqrt{4}=2$ que no es un rectángulo dinámico, $AUNC$ denotado por $\sqrt{5}$ y los rectángulos relacionado con el numero de oro $ACPP'$ denotado por $\phi$ construido con la ayuda del punto medio $D$ del segmento $OA$ finalmente el rectángulo $Orr'C$ denotado por $\sqrt{\phi}$ son los rectángulos más interesantes para la distribución de los elementos en el espacio bidimensional.



Se descompondrá armónicamente cada uno de estos rectángulos, saber el procedimiento es muy útil para los artistas plásticos sobre para los pintores en sus diversas composiciones bidimensionales, para aquellos que tienen noción tridimensional  se trataran de solidos en el siguiente capitulo.

\begin{comen}\label{com1} un rectángulo esta bien representado por su diagonal y la pendiente de esta en un sistema de ejes coordenados usual. Pues si tratamos de averiguar el tipo de rectángulo  lo que se hace es verificar  la razón de la longitud de su lado mayor y al longitud de su lado menor es decir, la pendiente de la diagonal con respecto aun sistema de ejes coordenados donde el eje las $x$ coincide con el lado mayor es decir en la Figura \ref{Op} la pendiente de la diagonal $AC$ es $\tan{\alpha}=\frac{\overline{CB}}{\overline{AB}}.$

	Por ejemplo en la Figura \ref{Up} el rectángulo $A'B'C'D'$ tiene las mismas proporciones que $ABCD$ pues la pendiente de $A'C'$ es la misma que la pendiente de $AC,$ este principio nos ayudara a demostrar algunas propiedades de los rectángulos dinámicos.
	\begin{figure}[!ht]
		\begin{center}

		\end{center}
		\caption{Tipo de rectángulo}\label{Op}
	\end{figure}


\end{comen}



\begin{comen}
	A partir de ahora se se usará la notación $ABCD\sim r, r\in \mathbb{I}$ donde $ABCD$ es un rectángulo y ''$\sim$'' significa ''similar semejante'', muy útil  para denotar que dos rectángulos tiene las mismas proporciones o la misma razón entre las longitudes de sus lados  por ejemplo en la Figura \ref{Up} $A'B'C'D'\sim\frac{B'C'}{A'B'}=k; k\in \mathbb{I}$ o en la Figura \ref{Uk} se tiene que $OABC\sim\frac{AB}{OA}=1.$

	\begin{figure}[!ht]
		\begin{center}
		\end{center}
		\caption{Cuadrado}\label{Up}
	\end{figure}

\end{comen}

\begin{comen}

	El siguiente criterio mostrada en la Figura \ref{Upu} se toma el $M=\frac{BC}{2}$ se traza el arco $CB$ centrada en $M$ luego $P$ es la intersección de la diagonal del rectángulo $ABCD$ con éste arco, finalmente $Q$ es la intersección del lado $DC$ con la linea $BP$. Se usara este principio para resumir las demostraciones de las propiedades de los rectángulos dinámicos, se tiene que $ABCD\sim P'BCQ$ pues en $AC\perp BQ$ esto es $\angle{BAC}=\angle{CBQ},$
	luego segun el Comentario \ref{com1} se tiene que $ABCD\sim P'BCQ$ y tambien se tiene que $\frac{AB}{CB}=\frac{BC}{QC}\Longleftrightarrow QC=\frac{BC^2}{AB}$ si $BC=1$ se tiene que $QC=\frac{1}{AB}$ por lo que si $AB$ es de la forma $\sqrt{\beta}$ se tiene que $QC=\frac{1}{\sqrt{\beta}}=\frac{\sqrt{\beta}}{\beta}$ es decir $QC=\frac{AB}{\beta},$ como un ejemplo particular se tiene que si $AB=\sqrt{6}\Longrightarrow QC=\frac{AB}{6}.$

	\begin{figure}[!ht]
		\begin{center}

		\end{center}
		\caption{Un rectángulo arbitrario}\label{Upu}
	\end{figure}

\end{comen}



\subsection{El cuadrado}
Para poder particionarlo es necesario hallar la sección áurea en uno de los lados por ejemplo $P$ con el método ya aprendido, a partir de allí se generan infinidad de posibilidades  por ejemplo una de ellas es la que se muestra en la figura siguiente. aunque el cuadrado es considerado menos apto para las composiciones con un poco de subdivisiones armónicas se pueden obtener una buena composición

\begin{figure}[!ht]
	\begin{center}

	\end{center}
	\caption{Cuadrado}\label{Uk}
\end{figure}

EL cuadrado suele ser uno de los formatos menos eficientes debido a su alta simetría pero con particiones adecuadas sobre su superficie se puede lograr grandes objetivos


\subsection{El rectángulo $\sqrt{2}$}

Siendo $M'$ y $M$ puntos medios de $DC$ y $AB$ se observa la propiedad de $DM\perp AC$  pues la pendiente del a recta $DM$ es $-\frac{2}{\sqrt{2}}$ y la pendiente de la recta $AC$ es $\frac{\sqrt{2}}{2}$ lo cual al multiplicar estas pendientes resulta $-1.$

Otra característica es que $ONMB$  es otro rectángulo $\sqrt{2}$ con el lado mayor $ON=MB$ pues $OM=\frac{\sqrt{2}}{2}$ y $NB=\frac{1}{2}$ entonces $\frac{AM}{NB}=\frac{\frac{\sqrt{2}}{2}}{\frac{1}{2}}=\frac{1}{2},$ en este rectángulo también se observa que $MH\perp HB$ pues $MC$ lo secciona a $ON$ en dos segmentos iguales $OP=PN$ lo cual usando el mismo criterio para el aso anterior  se verifica que $MH\perp HB,$ $HH''\perp H''B$ y $N'N\perp HB$ porque estos puntos se obtiene con el mismo procedimiento.

Finalmente se pueden obtener de manera indefinida rectángulos $\sqrt{2}$ tales como $OMNB,$ $HPBN',$ etc. los cuales convergen hacia el vértice $B.$ También se los puede hacer converger hacia los demás vertices $A, D$ y $C$ del mismo modo en el rectángulo $MONB$ se puede iterar los procedimientos anteriores hacia el infinito.

\begin{figure}[!ht]
	\begin{center}

	\end{center}
	\caption{Rectángulo $\sqrt{2}$}\label{de}
\end{figure}

\subsection{El rectángulo $\sqrt{3}$}


La propiedad de este triángulo es que si lo dividimos en tres franjas verticales iguales tales como $ANOD$,$ONMP$ y $PMBC$ obtenemos otros triángulos semejantes al primero $\sqrt{3}$, como en el caso anterior se uso las pendientes para averiguar si es correcto poner los ángulos rectos donde lo están, luego es posible iterar esta operación al infinito sobre cada uno de los tres rectángulos  obtenidos anteriormente para obtener otros con la misma propiedad pero en escala menor.


\begin{figure}[!ht]
	\begin{center}

	\end{center}
	\caption{Rectángulo $\sqrt{3}$}\label{3}
\end{figure}

pues $ANOD,$ $AN=\frac{\sqrt{3}}{3}$ y $ON=1$ luego $\frac{ON}{AN}=\frac{1}{\frac{\sqrt{3}}{3}}=\sqrt{3},$ esto es valido para $ONMP,$ $PMBC$ pues $ON=PM=CB$ y $AN=NM=MB.$ Para verificar que $DH\perp AC$ se tiene

\subsection{El rectángulo $\sqrt{5}$}

En este rectángulo se incluye los rectángulos $\phi$ y $\sqrt{5}$  como se muestra en la figura el rectángulo $A'BCD'$ y $AB'C'D$ son rectángulos $\phi$

Se empieza construyendo un cuadrado $A'B'C'D'$ al tomemos uno de sus lados $A'B'$ divisándolo en dos segmentos iguales $A'M=MB'$ el arco generado por $MC'$ interseca a la proyección de lado $A'B'$ en los dos puntos $A$ y $B$ observe que se utilizo el mismo procedimiento para obtener el rectángulo áureo pero en este caso se obtiene dos rectángulos áureos intersecando que comparten el mismo cuadrado, observe que $A'C\perp BC'$ entonces el arco $CB$ pasa por la intersecion de esta lineas, la diagonal $AC$ pasa por la interseccion de los arcos $AB$ y $CB$

\begin{figure}[!ht]
	\begin{center}\psset{unit=1.2cm}

\end{center}
	\caption{Rectángulo $\sqrt{5}$}\label{Uw}
\end{figure}
se obtiene $G$ intersecar el arco $BC$ con la diagonal $AC,$ $H$ al intersecar el lado $DC$ con la proyección de $BG$ finalmente $H'$ e $I$ al proyectar $H$ perpendicularmente sobre el lado $AB;$  sabe que el rectángulo $OO'CC'$ es un rectángulo $\qw$ por lo tanto $IH''CC'$ lo es, pues la diagonal $IC$ coincide con la del rectángulo $OO'CC'$ (Se demostró al principio de este capitulo que un rectángulo esta definido por el valor de la pendiente de su diagonal) se verifica que $HC=\frac{1}{5}DC$ entonces $H'BCH$ es un rectángulo $\sqrt{5},$ entonces como $H'BQJ'=1,$ y $IH''CH=\qw$  se deduce que $J'QH''I=\qw.$ También $EH'''CC'$ es un $\sqrt{5}$ pues comparten la diagonal del generador $ABCD,$ $BJ'$ es un cuadrado pues comparten la diagonal del cuadrado $B'BO'O$

\subsection{El rectángulo $\sqrt{\phi}$}

El rectángulo $\sqrt{\Phi}$ $AE'ED$ se obtiene a partir de un rectángulo áureo $\Phi$ como se muestra en la figura el rectángulo $\Phi,$ $ABCD$ se obtiene al trazar el arco $BE$  interceptando la linea $AC$ en $E,$ proyectando perpendicularmente este punto sobre la linea $AB$ se obtiene el cuarto vértice $E'$ del rectángulo $\sqrt{\phi}$
pues como se puede verificar se tiene que ${AE'}^2=\phi^2-1=2\phi+1-1=\phi\Longleftrightarrow AE'=\sqrt{\phi}$
%https://asy.marris.fr/#section3
\begin{figure}[!ht]
	\begin{center}

	\end{center}

	\caption{Rectángulo $\sqrt{\phi}$}\label{p}
\end{figure}







\subsection{El rectángulo áureo ($\phi$)}
El rectángulo de la siguiente figura tiene la única propiedad que si nosotros construimos un cuadrado sobre su lado pequeño(el menor termino del radio $\qw$), el rectángulo pequeño $aBCd$ formado a lado de este cuadrado el rectángulo original también es rectángulo $\qw,$ similar al primero. Esta operación puede ser repetido indefinidamente,ente, resultando así que los cuadrado pequeños, y pequeños y pequeños rectángulos áureos (la superficie del cuadrado y la superficie de los rectángulos formado geométricamente proverbio decreciente de radio $\frac{1}{\qw^2}$), como en la Figura ... Aun aun que actualmente dibujando el cuadrado , esta operación  y la proporción continua característica de la serie de los segmentos y superficies correlacionadas son subsecuentemente subconscientes  al ojo ; lo importante de esta operación  sugerente en la simple caso de una linea recta en dos segmentos  de acuerdo  a la sección áurea


\begin{figure}[!ht]
	\begin{center}
	\end{center}
	\caption{Rectángulo $\phi$}\label{Uww}
\end{figure}



\subsection{El triángulo áureo y el pentágono}
Es fácil verificar los ángulos mostrados en la figura pues los ángulos interiores de un pentágono son  como $EDC=180^\circ-\frac{360^\circ}{5}=108$ luego el ángulo $DEC=DCE=\frac{180^\circ-108^\circ}{2}=36^\circ$  ya que $I'D=I'E$ se deduce que el ángulo $DI'I''=72^\circ.$

Se prueba fácilmente que $\frac{a}{b}=\phi$ pues usando la ley de los senos en el triangulo $ADB$ se tiene que: $$\frac{a}{\sin72}=\frac{b}{\sin36}\Longleftrightarrow \frac{a}{b}=\frac{\sin72}{\sin36}=\frac{0.95105651629515357211\ldots}{0.5877852522924731291\ldots}=\phi.$$

Lo mismo ocurre con $\frac{DI'}{I'I}=\phi$ pues solo basta probar que los segmentos $I'I$ y $I'I''$ son iguales en efecto pues $I'I''$ es el lado del pentágono que se genera con la diagonales del pentágono $ABCDE$


\begin{figure}[!ht]
	\begin{center}

	\end{center}

	\caption{El Pentágono y el Triángulo Áureo y la Relación de sus Lados}\label{R}
\end{figure}


\subsection{Ejemplos de Composición sobre los Rectángulos Dinámicos}


\begin{longtable}{|l|l|l|}
	\caption{A sample long table.} \label{tab:long} \\

	\hline \multicolumn{1}{|c|}{\textbf{First column}} & \multicolumn{1}{c|}{\textbf{Second column}} & \multicolumn{1}{c|}{\textbf{Third column}} \\ \hline
	\endfirsthead

	\multicolumn{3}{c}%
	{{\bfseries \tablename\ \thetable{} -- continued from previous pagewwwwww}} \\
	\hline \multicolumn{1}{|c|}{\textbf{First column}} & \multicolumn{1}{c|}{\textbf{Second column}} & \multicolumn{1}{c|}{\textbf{Third column}} \\ \hline
	\endhead

	\hline \multicolumn{3}{|r|}{{Continued on next pagewwwwwwwwwwwwwww}} \\ \hline
	\endfoot

	\hline
	\endlastfoot

	One & abcdef ghjijklmn & 123.456778 \\
	One & abcdef ghjijklmn & 123.456778 \\
	One & abcdef ghjijklmn & 123.456778 \\
	One & abcdef ghjijklmn & 123.456778 \\
	One & abcdef ghjijklmn & 123.456778 \\
	One & abcdef ghjijklmn & 123.456778 \\
	One & abcdef ghjijklmn & 123.456778 \\
	One & abcdef ghjijklmn & 123.456778 \\
	One & abcdef ghjijklmn & 123.456778 \\
	One & abcdef ghjijklmn & 123.456778 \\
	One & abcdef ghjijklmn & 123.456778 \\
	One & abcdef ghjijklmn & 123.456778 \\
	One & abcdef ghjijklmn & 123.456778 \\
	One & abcdef ghjijklmn & 123.456778 \\
	One & abcdef ghjijklmn & 123.456778 \\
	One & abcdef ghjijklmn & 123.456778 \\
	One & abcdef ghjijklmn & 123.456778 \\
	One & abcdef ghjijklmn & 123.456778 \\
	One & abcdef ghjijklmn & 123.456778 \\
	One & abcdef ghjijklmn & 123.456778 \\
	One & abcdef ghjijklmn & 123.456778 \\
	One & abcdef ghjijklmn & 123.456778 \\
	One & abcdef ghjijklmn & 123.456778 \\
	One & abcdef ghjijklmn & 123.456778 \\
	One & abcdef ghjijklmn & 123.456778 \\
	One & abcdef ghjijklmn & 123.456778 \\
	One & abcdef ghjijklmn & 123.456778 \\
	One & abcdef ghjijklmn & 123.456778 \\
	One & abcdef ghjijklmn & 123.456778 \\
	One & abcdef ghjijklmn & 123.456778 \\
	One & abcdef ghjijklmn & 123.456778 \\
	One & abcdef ghjijklmn & 123.456778 \\
	One & abcdef ghjijklmn & 123.456778 \\
	One & abcdef ghjijklmn & 123.456778 \\
	One & abcdef ghjijklmn & 123.456778 \\
	One & abcdef ghjijklmn & 123.456778 \\
	One & abcdef ghjijklmn & 123.456778 \\
	One & abcdef ghjijklmn & 123.456778 \\
	One & abcdef ghjijklmn & 123.456778 \\
	One & abcdef ghjijklmn & 123.456778 \\
	One & abcdef ghjijklmn & 123.456778 \\
	One & abcdef ghjijklmn & 123.456778 \\
	One & abcdef ghjijklmn & 123.456778 \\
	One & abcdef ghjijklmn & 123.456778 \\
	One & abcdef ghjijklmn & 123.456778 \\
	One & abcdef ghjijklmn & 123.456778 \\
	One & abcdef ghjijklmn & 123.456778 \\
	One & abcdef ghjijklmn & 123.456778 \\
	One & abcdef ghjijklmn & 123.456778 \\
	One & abcdef ghjijklmn & 123.456778 \\
	One & abcdef ghjijklmn & 123.456778 \\
	One & abcdef ghjijklmn & 123.456778 \\
	One & abcdef ghjijklmn & 123.456778 \\
	One & abcdef ghjijklmn & 123.456778 \\
	One & abcdef ghjijklmn & 123.456778 \\
	One & abcdef ghjijklmn & 123.456778 \\
	One & abcdef ghjijklmn & 123.456778 \\
	One & abcdef ghjijklmn & 123.456778 \\
	One & abcdef ghjijklmn & 123.456778 \\
	One & abcdef ghjijklmn & 123.456778 \\
	One & abcdef ghjijklmn & 123.456778 \\
	One & abcdef ghjijklmn & 123.456778 \\
	One & abcdef ghjijklmn & 123.456778 \\
	One & abcdef ghjijklmn & 123.456778 \\
	One & abcdef ghjijklmn & 123.456778 \\
	One & abcdef ghjijklmn & 123.456778 \\
	One & abcdef ghjijklmn & 123.456778 \\
	One & abcdef ghjijklmn & 123.456778 \\
	One & abcdef ghjijklmn & 123.456778 \\
	One & abcdef ghjijklmn & 123.456778 \\
	One & abcdef ghjijklmn & 123.456778 \\
	One & abcdef ghjijklmn & 123.456778 \\
	One & abcdef ghjijklmn & 123.456778 \\
	One & abcdef ghjijklmn & 123.456778 \\
	One & abcdef ghjijklmn & 123.456778 \\
	One & abcdef ghjijklmn & 123.456778 \\
	One & abcdef ghjijklmn & 123.456778 \\
	One & abcdef ghjijklmn & 123.456778 \\
	One & abcdef ghjijklmn & 123.456778 \\
	One & abcdef ghjijklmn & 123.456778 \\
\end{longtable}
