\chapter{Fractales}

\section{Fractales 2D}


Un conjunto $E\subset \Real^n$ es autosemejante si existe una coleccion $\xi_1,\xi_2 \cdots \xi_m$ de semejansas de $\Real^n,$ todoas ellas con razones menores a la unidad (es decri contractivas), tales que

\begin{itemize}
	\item $E\subset \bigcup^m_{i=1}\xi_i\pa{E}$
	\item para cierrto $s$ (no necesariamente entero) se tiene ue $H^s\pa{E}>0$ y que $H^s\pa{\xi_i\pa{E}\cap \xi\pa{E}}=0,$ si $i\neq j$
\end{itemize}

Las estructuras de las superficies que observamos no son lo que parecen con una finitud limitada mas aun se puede sumergir al infinito atómicamente y averiguar mentalmente la composición estructural de tales formas, desde  la concepción de este tema e podrá cambiar el modo de ver de las cosas .

\subsection{Phylotaxis}
The beautiful arrangement de las hojas en alguna plantas, llamado phyllotaxis, obeys un munero de subtle mathematical relationships. For instance, the florets in the head of a sunflower form two oppositely directed spirals: 55 of them clockwise and 34 counterclockwise.Surprisingly, these numbers are consecutive Fibonacci numbers. The Phyllotaxis is like a Lindenmayer system.


\begin{figure}[!ht]
	\begin{center}
\begin{asy}
size(300,0);
import markers;
import geometry;
import math;
pair A=0, B=(1,0), C=(0.7,1), D=(-0.5,0), F=rotate(-90)*(C-B)/2+B;
draw(A--B);
draw(A--C);
pen p=linewidth(1mm);
draw(B--C,p);
draw(A--D);
draw(B--F,p);
label("$A$",A,SW);
label("$B$",B,S);
label("$C$",C,N);
dot(Label("$D$",D,S));
dot(Label("$F$",F,N+NW));
markangle(A,C,B);
markangle(scale(1.5)*"$\theta$",radius=40,C,B,A,ArcArrow(2mm),1mm+red);
markangle(scale(1.5)*"$-\theta$",radius=-70,A,B,C,ArcArrow,green);
markangle(Label("$\gamma$",Relative(0.25)),n=2,radius=-30,A,C,B,p=0.7blue+2);
markangle(n=3,B,A,C,marker(markinterval(stickframe(n=2),true)));
pen RedPen=0.7red+1bp;
markangle(C,A,D,RedPen,marker(markinterval(2,stickframe(3,4mm,RedPen),true)));
drawline(A,A+dir(A--D,A--C),dotted);
perpendicular(B,NE,F-B,size=10mm,1mm+red,
TrueMargin(linewidth(p)/2,linewidth(p)/2),Fill(yellow));
\end{asy}

	\end{center}
	\caption{e}
\end{figure}


\subsection{Cosh y sierpinsiqui}

\begin{figure}[!ht]
	\begin{center}
\begin{asy}
size(300,0);
import markers;
import geometry;
import math;
pair A=0, B=(1,0), C=(0.7,1), D=(-0.5,0), F=rotate(-90)*(C-B)/2+B;
draw(A--B);
draw(A--C);
pen p=linewidth(1mm);
draw(B--C,p);
draw(A--D);
draw(B--F,p);
label("$A$",A,SW);
label("$B$",B,S);
label("$C$",C,N);
dot(Label("$D$",D,S));
dot(Label("$F$",F,N+NW));
markangle(A,C,B);
markangle(scale(1.5)*"$\theta$",radius=40,C,B,A,ArcArrow(2mm),1mm+red);
markangle(scale(1.5)*"$-\theta$",radius=-70,A,B,C,ArcArrow,green);
markangle(Label("$\gamma$",Relative(0.25)),n=2,radius=-30,A,C,B,p=0.7blue+2);
markangle(n=3,B,A,C,marker(markinterval(stickframe(n=2),true)));
pen RedPen=0.7red+1bp;
markangle(C,A,D,RedPen,marker(markinterval(2,stickframe(3,4mm,RedPen),true)));
drawline(A,A+dir(A--D,A--C),dotted);
perpendicular(B,NE,F-B,size=10mm,1mm+red,
TrueMargin(linewidth(p)/2,linewidth(p)/2),Fill(yellow));
\end{asy}

	\end{center}
	\caption{s}\label{s}
\end{figure}

\begin{figure}[!ht]
	\begin{center}
\begin{asy}
size(300,0);
import markers;
import geometry;
import math;
pair A=0, B=(1,0), C=(0.7,1), D=(-0.5,0), F=rotate(-90)*(C-B)/2+B;
draw(A--B);
draw(A--C);
pen p=linewidth(1mm);
draw(B--C,p);
draw(A--D);
draw(B--F,p);
label("$A$",A,SW);
label("$B$",B,S);
label("$C$",C,N);
dot(Label("$D$",D,S));
dot(Label("$F$",F,N+NW));
markangle(A,C,B);
markangle(scale(1.5)*"$\theta$",radius=40,C,B,A,ArcArrow(2mm),1mm+red);
markangle(scale(1.5)*"$-\theta$",radius=-70,A,B,C,ArcArrow,green);
markangle(Label("$\gamma$",Relative(0.25)),n=2,radius=-30,A,C,B,p=0.7blue+2);
markangle(n=3,B,A,C,marker(markinterval(stickframe(n=2),true)));
pen RedPen=0.7red+1bp;
markangle(C,A,D,RedPen,marker(markinterval(2,stickframe(3,4mm,RedPen),true)));
drawline(A,A+dir(A--D,A--C),dotted);
perpendicular(B,NE,F-B,size=10mm,1mm+red,
TrueMargin(linewidth(p)/2,linewidth(p)/2),Fill(yellow));
\end{asy}

	\end{center}
	\caption{copo de nieva de Cosh}
\end{figure}






\subsection{Arboles}

\begin{figure}[!ht]
	\begin{center}
\begin{asy}
size(300,0);
import markers;
import geometry;
import math;
pair A=0, B=(1,0), C=(0.7,1), D=(-0.5,0), F=rotate(-90)*(C-B)/2+B;
draw(A--B);
draw(A--C);
pen p=linewidth(1mm);
draw(B--C,p);
draw(A--D);
draw(B--F,p);
label("$A$",A,SW);
label("$B$",B,S);
label("$C$",C,N);
dot(Label("$D$",D,S));
dot(Label("$F$",F,N+NW));
markangle(A,C,B);
markangle(scale(1.5)*"$\theta$",radius=40,C,B,A,ArcArrow(2mm),1mm+red);
markangle(scale(1.5)*"$-\theta$",radius=-70,A,B,C,ArcArrow,green);
markangle(Label("$\gamma$",Relative(0.25)),n=2,radius=-30,A,C,B,p=0.7blue+2);
markangle(n=3,B,A,C,marker(markinterval(stickframe(n=2),true)));
pen RedPen=0.7red+1bp;
markangle(C,A,D,RedPen,marker(markinterval(2,stickframe(3,4mm,RedPen),true)));
drawline(A,A+dir(A--D,A--C),dotted);
perpendicular(B,NE,F-B,size=10mm,1mm+red,
TrueMargin(linewidth(p)/2,linewidth(p)/2),Fill(yellow));
\end{asy}

	\end{center}
	\caption{Arbol}\label{f}
\end{figure}

\begin{figure}[!ht]
	\begin{center}
\begin{asy}
size(300,0);
import markers;
import geometry;
import math;
pair A=0, B=(1,0), C=(0.7,1), D=(-0.5,0), F=rotate(-90)*(C-B)/2+B;
draw(A--B);
draw(A--C);
pen p=linewidth(1mm);
draw(B--C,p);
draw(A--D);
draw(B--F,p);
label("$A$",A,SW);
label("$B$",B,S);
label("$C$",C,N);
dot(Label("$D$",D,S));
dot(Label("$F$",F,N+NW));
markangle(A,C,B);
markangle(scale(1.5)*"$\theta$",radius=40,C,B,A,ArcArrow(2mm),1mm+red);
markangle(scale(1.5)*"$-\theta$",radius=-70,A,B,C,ArcArrow,green);
markangle(Label("$\gamma$",Relative(0.25)),n=2,radius=-30,A,C,B,p=0.7blue+2);
markangle(n=3,B,A,C,marker(markinterval(stickframe(n=2),true)));
pen RedPen=0.7red+1bp;
markangle(C,A,D,RedPen,marker(markinterval(2,stickframe(3,4mm,RedPen),true)));
drawline(A,A+dir(A--D,A--C),dotted);
perpendicular(B,NE,F-B,size=10mm,1mm+red,
TrueMargin(linewidth(p)/2,linewidth(p)/2),Fill(yellow));
\end{asy}

	\end{center}
	\caption{e}
\end{figure}

\begin{figure}[!ht]
	\begin{center}
\begin{asy}
size(300,0);
import markers;
import geometry;
import math;
pair A=0, B=(1,0), C=(0.7,1), D=(-0.5,0), F=rotate(-90)*(C-B)/2+B;
draw(A--B);
draw(A--C);
pen p=linewidth(1mm);
draw(B--C,p);
draw(A--D);
draw(B--F,p);
label("$A$",A,SW);
label("$B$",B,S);
label("$C$",C,N);
dot(Label("$D$",D,S));
dot(Label("$F$",F,N+NW));
markangle(A,C,B);
markangle(scale(1.5)*"$\theta$",radius=40,C,B,A,ArcArrow(2mm),1mm+red);
markangle(scale(1.5)*"$-\theta$",radius=-70,A,B,C,ArcArrow,green);
markangle(Label("$\gamma$",Relative(0.25)),n=2,radius=-30,A,C,B,p=0.7blue+2);
markangle(n=3,B,A,C,marker(markinterval(stickframe(n=2),true)));
pen RedPen=0.7red+1bp;
markangle(C,A,D,RedPen,marker(markinterval(2,stickframe(3,4mm,RedPen),true)));
drawline(A,A+dir(A--D,A--C),dotted);
perpendicular(B,NE,F-B,size=10mm,1mm+red,
TrueMargin(linewidth(p)/2,linewidth(p)/2),Fill(yellow));
\end{asy}

	\end{center}
	\caption{triangulo de Sierpinski}
\end{figure}


\subsection{Circulo de Apollonius}

\begin{figure}[!ht]
	\begin{center}
\begin{asy}
size(300,0);
import markers;
import geometry;
import math;
pair A=0, B=(1,0), C=(0.7,1), D=(-0.5,0), F=rotate(-90)*(C-B)/2+B;
draw(A--B);
draw(A--C);
pen p=linewidth(1mm);
draw(B--C,p);
draw(A--D);
draw(B--F,p);
label("$A$",A,SW);
label("$B$",B,S);
label("$C$",C,N);
dot(Label("$D$",D,S));
dot(Label("$F$",F,N+NW));
markangle(A,C,B);
markangle(scale(1.5)*"$\theta$",radius=40,C,B,A,ArcArrow(2mm),1mm+red);
markangle(scale(1.5)*"$-\theta$",radius=-70,A,B,C,ArcArrow,green);
markangle(Label("$\gamma$",Relative(0.25)),n=2,radius=-30,A,C,B,p=0.7blue+2);
markangle(n=3,B,A,C,marker(markinterval(stickframe(n=2),true)));
pen RedPen=0.7red+1bp;
markangle(C,A,D,RedPen,marker(markinterval(2,stickframe(3,4mm,RedPen),true)));
drawline(A,A+dir(A--D,A--C),dotted);
perpendicular(B,NE,F-B,size=10mm,1mm+red,
TrueMargin(linewidth(p)/2,linewidth(p)/2),Fill(yellow));
\end{asy}

	\end{center}
	\caption{Circulo de Apollonius}
\end{figure}

\begin{figure}[!ht]
	\begin{center}
\begin{asy}
size(300,0);
import markers;
import geometry;
import math;
pair A=0, B=(1,0), C=(0.7,1), D=(-0.5,0), F=rotate(-90)*(C-B)/2+B;
draw(A--B);
draw(A--C);
pen p=linewidth(1mm);
draw(B--C,p);
draw(A--D);
draw(B--F,p);
label("$A$",A,SW);
label("$B$",B,S);
label("$C$",C,N);
dot(Label("$D$",D,S));
dot(Label("$F$",F,N+NW));
markangle(A,C,B);
markangle(scale(1.5)*"$\theta$",radius=40,C,B,A,ArcArrow(2mm),1mm+red);
markangle(scale(1.5)*"$-\theta$",radius=-70,A,B,C,ArcArrow,green);
markangle(Label("$\gamma$",Relative(0.25)),n=2,radius=-30,A,C,B,p=0.7blue+2);
markangle(n=3,B,A,C,marker(markinterval(stickframe(n=2),true)));
pen RedPen=0.7red+1bp;
markangle(C,A,D,RedPen,marker(markinterval(2,stickframe(3,4mm,RedPen),true)));
drawline(A,A+dir(A--D,A--C),dotted);
perpendicular(B,NE,F-B,size=10mm,1mm+red,
TrueMargin(linewidth(p)/2,linewidth(p)/2),Fill(yellow));
\end{asy}

	\end{center}
	\caption{Circulo de Apollonius}
\end{figure}




\section{Fractales 3D}


Fractales matemáticos en tres dimensiones. Introducción a los fractales. La geometría fractal estudia las formas que tienen dimensión fraccionaria.


\begin{figure}[!ht]
	\centering
	\begin{asy}
	size(200);
	import palette;
	import three;

	currentprojection=perspective(1,1,1);

	triple[] M=
	{
	(-1,-1,-1),(0,-1,-1),(1,-1,-1),(1,0,-1),
	(1,1,-1),(0,1,-1),(-1,1,-1),(-1,0,-1),
	(-1,-1,0),(1,-1,0),(1,1,0),(-1,1,0),
	(-1,-1,1),(0,-1,1),(1,-1,1),(1,0,1),(1,1,1),(0,1,1),(-1,1,1),(-1,0,1)
	};

	surface[] Squares=
	{
	surface((1,-1,-1)--(1,1,-1)--(1,1,1)--(1,-1,1)--cycle),
	surface((-1,-1,-1)--(-1,1,-1)--(-1,1,1)--(-1,-1,1)--cycle),
	surface((1,1,-1)--(-1,1,-1)--(-1,1,1)--(1,1,1)--cycle),
	surface((1,-1,-1)--(-1,-1,-1)--(-1,-1,1)--(1,-1,1)--cycle),
	surface((1,-1,1)--(1,1,1)--(-1,1,1)--(-1,-1,1)--cycle),
	surface((1,-1,-1)--(1,1,-1)--(-1,1,-1)--(-1,-1,-1)--cycle),
	};

	int[][] SquaresPoints=
	{
	{2,3,4,10,16,15,14,9},
	{0,7,6,11,18,19,12,8},
	{4,5,6,11,18,17,16,10},
	{2,1,0,8,12,13,14,9},
	{12,13,14,15,16,17,18,19},
	{0,1,2,3,4,5,6,7}
	};

	int[][] index=
	{
	{0,2,4},{0,1},{1,2,4},{2,3},{1,3,4},{0,1},{0,3,4},{2,3},
	{4,5},{4,5},{4,5},{4,5},
	{0,2,5},{0,1},{1,2,5},{2,3},{1,3,5},{0,1},{0,3,5},{2,3}
	};

	int[] Sponge0=array(n=6,value=1);

	int[] eraseFaces(int n, int[] Sponge0) {
	int[] temp=copy(Sponge0);
	for(int k : index[n]) {
	temp[k]=0;
	}
	return temp;
	}

	int[][] Sponge1=new int[20][];
	for(int n=0; n < 20; ++n) {
	Sponge1[n]=eraseFaces(n,Sponge0);
	}

	int[][] eraseFaces(int n, int[][] Sponge1) {
	int[][] temp=copy(Sponge1);
	for(int k : index[n])
	for(int n1 : SquaresPoints[k])
	temp[n1][k]=0;
	return temp;
	}

	int[][][] Sponge2=new int[20][][];
	for(int n=0; n < 20; ++n)
	Sponge2[n]=eraseFaces(n,Sponge1);

	int[][][] eraseFaces(int n, int[][][] Sponge2) {
	int[][][] temp=copy(Sponge2);
	for(int k : index[n])
	for(int n2: SquaresPoints[k])
	for(int n1: SquaresPoints[k])
	temp[n2][n1][k]=0;
	return temp;
	}

	int[][][][] Sponge3=new int[20][][][];
	for(int n=0; n < 20; ++n)
	Sponge3[n]=eraseFaces(n,Sponge2);

	surface s3;
	real u=2/3;
	for(int n3=0; n3 < 20; ++n3) {
	surface s2;
	for(int n2=0; n2 < 20; ++n2) {
	surface s1;
	for(int n1=0; n1 < 20; ++n1) {
	for(int k=0; k < 6; ++k) {
	if(Sponge3[n3][n2][n1][k] > 0) {
	s1.append(scale3(u)*shift(M[n1])*scale3(0.5)*Squares[k]);
	}
	}
	}
	s2.append(scale3(u)*shift(M[n2])*scale3(0.5)*s1);
	}
	s3.append(scale3(u)*shift(M[n3])*scale3(0.5)*s2);
	}
	s3.colors(palette(s3.map(abs),Rainbow()));
	draw(s3);
	\end{asy}

	\caption{Esponja de Menger}
\end{figure}


En matemáticas, la esponja de Menger (a veces llamada cubo de Menger o bien cubo o esponja de Menger-Sierpinski o de Sierpiński) es un conjunto fractal descrito por primera vez en 1926 por Karl Menger mientras exploraba el concepto de dimensión topológica

Al igual que la alfombra de Sierpinski constituye una generalización bidimensional del conjunto de Cantor, esta es una generalización tridimensional de ambos. Comparte con estos muchas de sus propiedades, siendo un conjunto compacto, no numerable y de medida de Lebesgue nula. Su dimensión dimensión fractal de Hausdorff es ${ d_{H}=\log 20/\log 3\approx 2.7268}d_{H}=\log 20/\log 3\approx 2.7268$. El área de la esponja de Menger es infinita y al mismo tiempo encierra un volumen cero.




Es de destacar su propiedad de curva universal, pues es un conjunto topológico de dimensión topológica uno, y cualquier otra curva o grafo es homeomorfo a un subconjunto de la esponja de Menger.


La construcción de la esponja de Menger se define de forma recursiva:

\begin{enumerate}
	\item Comenzamos con un cubo (primera imagen).
	\item Dividimos cada cara del cubo en 9 cuadrados. Esto subdivide el cubo en 27 cubos más pequeños, como le sucede al cubo de Rubik.
	\item Eliminamos los cubos centrales de cada cara (6) y el cubo central (1), dejando solamente 20 cubos (segunda imagen).
	\item Repetimos los pasos 1, 2 y 3 para cada uno de los veinte cubos menores restantes.
\end{enumerate}
La esponja de Menger es el límite de este proceso tras un número infinito de iteraciones.


Karl Menger (Viena, Austria, 13 de enero de 1902 - Highland Park, Illinois, EE.UU., 5 de octubre de 1985) fue un matemático, hijo del famoso economista Carl Menger, conocido por el teorema de Menger. Dentro de las matemáticas trabajó en álgebra, álgebra de la geometría, teoría de la curva y la dimensión, etc. Además, contribuyó a la teoría de juegos y a las ciencias sociales.
