\chapter{Canon Griego y Gótico}
El buen gusto por la perfección exquisitas llevo a los generadores de formas reales o abstractas a concebir cañones y reglas, la analogía de la sección áurea con muchas areas de la ciencia no explica como el numero del promedio entre el caos y el orden los rectángulos estáticos'' no producen divisiones armónicas, no obstante los rectángulos dinámicos producen las mas variadas y satisfactorias subdivisiones  y combinaciones distintas sin encontrar antagonismos entre ellos mas aun viéndolas unirse entre ellas para genera un solo objeto bidimensionales  por ejemplo el $\sqrt{5}$ se compone de muchos de el mismo y del rectángulo $\phi$





\begin{asy}
	 import graph;
size(300,150,IgnoreAspect);

real f(real x) {return x*sin(x);}
pair F(real x) {return (x,f(x));}

dotfactor=7;

void subinterval(real a, real b)
{
path g=box((a,0),(b,f(b)));
filldraw(g,orange+opacity(0.5));
draw(box((a,f(a)),(b,0)));
}

int a=-9, b=9;

xaxis("$x$",0,b);
yaxis("$y$",0);

draw(graph(f,a,b,n=200, Hermite));

int n=2;

for(int i=a; i <= b; ++i) {
if(i < b) subinterval(i,i+1);
if(i <= n) labelx(i);
dot(F(i));
}

int i=n;
labelx("$\ldots$",++i);
labelx("$k$",++i);
labelx("$k+1$",++i);
labelx("$\ldots$",++i);

//arrow("$f(x)$",F(i-1.5),NE,1.5cm,red,Margin(0,0.5));
\end{asy}
